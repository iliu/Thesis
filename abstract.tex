Cyber-Physical Systems (CPS) are integrations of computation with physical processes~\cite{Lee08_CyberPhysicalSystemsDesignChallenges}.
An number of applications can benefit from the potential of CPS. 
However, these systems must be equipped to handle the inherent concurrency and inexorable passage of time of physical processes.  
The traditional computing abstractions only concern themselves with the �functional� aspects of a program, and not its timing properties.
Thus, nearly every abstraction layer has failed to incorporate \emph{time} into its semantics; the passage of time is merely a consequence of the implementation. 
When the temporal properties of the system must be guaranteed, designers must reach beneath the abstraction layers.
This not only increases the design complexity and effort, but the designed systems are brittle and extremely sensitive to change.
In this work, we re-examine the ISA layer and its affects on microarchitecture design.
The ISA defines the contract between software instructions and hardware implementations. 
However, modern ISAs do not specify timing properties of instructions as part of the contract. 
Thus, architecture designs have largely implemented techniques that improve average performance at the expense of execution time variability.
This leads to imprecise WCET bounds that limit the timing predictability and timing composability of architectures.  

In order to address the lack of temporal semantics in the ISA, we propose instruction extensions to the ISA that give temporal meaning to the program. 
The instruction extensions allow programs to specify execution time properties in software that must be observed for any \emph{correct} execution of the program. 
In addition, we present the Precision Timed ARM (PTARM) architecture, a realization of Precision Timed (PRET) machines~\cite{edwards2007case} that provide timing predictability and composability without sacrificing performance. 
PTARM employs a predictable thread-interleaved pipeline with an exposed memory hierarchy that uses scratchpads and a predictable DRAM controller. 
This removes timing interference amongst the hardware threads, enabling timing composability in the architecture, and provides deterministic execution times for instructions within the architecture, enabling timing predictability in the architecture.  
We show that the predictable thread-interleaved pipeline and DRAM controller design also achieves better throughput compared to conventional architectures when fully utilized, accomplishing our goal to provide both predictability and performance.  
To show the applicability of the architecture, we present two applications implemented with the PRET architecture that utilize the predictable execution time and the timing extended ISA to achieve its design requirements.    
With this work, we aim to provide a deterministic foundation for higher abstraction layers, which enables more efficient designs of safety-critical cyber-physical systems.   
