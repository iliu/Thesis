\label{subsec:precision_timing_inst_ptarm}

%mention that a full scaled worst case execution time analysis is beyond the scope of this thesis.
%but do a brief summary on how it's done, cite wilhelm's group's research summary
Worst-case execution time (WCET) analysis is a combined analysis of the control paths that the software might exhibit, and the time it takes to execute those paths on the underlying architecture. 
A plethora of research has been done on the software analysis of control paths. 
Wilhelm et al.~\cite{wilhelm-survey-paper} presented a survey of tools and techniques available for worst-case path enumeration and loop analysis etc.
However, the precision of the WCET analysis of those techniques ultimately depend on the underlying architecture implementation~\cite{Heckmann2003processor}.
Architectures that exhibit wildly unpredictable execution times will result in overly conservative WCET analysis, even if the software structure is simple. 
Designed as a predictable architecture, the instructions of PTARM all exhibit deterministic timing behaviors, allowing a more precise WCET as software analysis progresses. 
In section~\ref{sec:ptarm_instructions} we discussed the implementation of each instruction and explained the execution time of each instruction type.
Table~\ref{table:ptarm_instruction_timing} summarizes the execution time each instruction takes in terms of \emph{thread cycles}. 

A \emph{Thread cycle} is the unit used to represent execution time on each thread.  
Timing analysis can be done separately for each hardware thread running on PTARM because the threads are temporally isolated; the execution time of each thread cannot be affected by other threads.
The thread-interleaved pipeline switches thread contexts every processor cycle in a predictable round robin fashion. 
Thus, each thread is fetched and executed in the pipeline every $N$ processor cycles, $N$ being the number of threads in the pipeline.
One \emph{Thread cycle} represents each time the thread enters in the pipeline, which is the thread's perceived notion of cycles.
The execution frequency of each thread ($F_{thread}$) is $F_{thread} = F_{processor}/N$, so each \emph{thread cycle} is $1/F_{thread}$ long. 
For example, our PTARM core is clocked at 100$MHz$ ($F_{processor} = 100 * 10^6$) and has 4 threads ($N=4$) , so each thread cycle is $1/((100 * 10^6)/4) = 40 * 10^{-9}$ secs, or 40 nanoseconds long.
The length of the \emph{thread cycle} will not change because of the predictable thread-switching policy, making it a reliable unit of measurement for execution time.   

\begin{table}
\noindent\makebox[\textwidth]{%
\begin{smalltabular}{ | p{4cm} | c || p{9cm} | }
  \hline                        
  \textbf{Instruction Type} & \textbf{Thread Cycles} & \textbf{Notes}    \\ \hline
  Data Processing  & 1 & \multirow{2}{9cm}{ ${\ast}$: an additional cycle is added if pre or post-index offset is used in the address mode.}  \\ \cline{1-2} 
  Branch  & 1 &  \\ \cline{1-2}
  Load/Store (SPM)  & 1$^{\ast}$ &  \multirow{4}{9cm}{ ${\phi}$: the dram latency is 3 or 4 thread cycles depending on the alignment of the pipeline and the dram controller backend, as described in section~\ref{sec:ptarm_dram_integration}. For conservative estimates, 4 thread cycles is used. } \\ \cline{1-2} 
  Load/Store (BootROM)  & 1$^{\ast}$ & \\ \cline{1-2}  
  Load/Store (DRAM) & 4$^{\phi \dagger \ast}$ & \\ \cline{1-2}
  Load/Store Multiple & $N_{reg} \times L_{mem}$ $^{\Delta \ast}$ &  \\  \cline{1-2}
  Software Interrupt (SWI) & 1 & \multirow{4}{9cm}{ ${\dagger}$: a single store buffer is implemented, as described in section~\ref{sec:ptarm_dram_store_buffer}, so non-consecutive stores take only 1 thread cycle, while consecutive stores require the full dram latency of 3 or 4.}\\    \cline{1-2}
  get\_time  & 2 &  \\ \cline{1-2}
  delay\_until  & 1 $^{\sigma}$ & \\    \cline{1-2}
  exception\_on\_expire  & 1  &    \\   \cline{1-2}
  deactivate\_exception & 1 &  \multirow{2}{9cm}{ ${\Delta}$: $N_{reg}$ is number of registers specified in the instruction. $L_{mem}$ is the latency of the memory region operated on.} \\  \cline{1-2}
  \multicolumn{2}{|c|}{} &  \\ \cline{1-2}
  \multicolumn{2}{|c|}{} &   \multirow{2}{9cm}{${\sigma}$: denotes the minimum execution time. Actual execution time depends on the system clock and the specified deadline.} \\ \cline{1-2}
  \multicolumn{2}{|c|}{} &\\ \hline
\end{smalltabular}}
\caption{Timing properties of PTARM instructions}
\label{table:ptarm_instruction_timing}
\end{table}

\subsection{Memory instructions}
Data-processing instructions typically have straightforward execution times on most architectures.
In our case, branch instructions also have very a predictable and straightforward execution, as the branch penalty is completely hidden within the thread interleaving.
The memory instructions in our architecture however can have several different latencies depending on addressing mode or region of access, as listed in table~\ref{table:ptarm_instruction_timing}.
For memory instructions that use pre or post-indexed addressing mode to update the base register, an additional cycle latency is needed to write back to the base register, as described in the instruction implementation in section~\ref{sec:ptarm_instruction_ldstr}.
But the addressing mode of load/store instructions can be determined statically, as it is part of the instruction encoding, so it does not affect the complexity or precision of execution time analysis. 

For load instructions, the exposed memory hierarchy allows us to clearly label and identify access latencies to different memory regions.
In execution time analysis, value analysis attempts to determine what memory addresses are accessed.     
In systems that use caches and hide the memory hierarchy, additional modeling of the cache state is still needed after the value analysis determines the potential memory addresses that are accessed.
However, with an exposed memory hierarchy with scratchpads, the precision of execution time analysis depends only on the value analysis. 
As soon as the memory address is determined, we can give a associate a precise memory access latency.
The worst-case DRAM latency of 4 thread cycles is derived from~\cite{ReinekeLiuPatelKimLee11_PRETDRAMControllerBankPrivatizationForPredictability}. 
This allows not only for a simpler timing analysis, but also a more accurate analysis.

For store instructions, the single store buffer as described in section~\ref{sec:ptarm_dram_store_buffer} can possibly hide the latency of store instructions if no other memory access instruction comes after it. 
Otherwise the store instruction will take the full memory access latency, depending on the memory access region of the store.
Timing analysis tools can mostly account for the store buffer by statically checking a few instructions ahead of the store instruction to see if there are memory accessing instructions.
The window of instructions that need to be checked is very narrow, so it should not really complicate the analysis.
If this is not possible, then conservatively, the full memory access latency should be used for each store, depending on the access region.

The execution time of load/store multiple instructions depend on the number of registers operated on, and the memory region it accesses. 
Because the register list is statically encoded in the instruction, the number of registers operated on can easily be statically determined.
Store multiple instructions do not benefit from the store buffer, so all accesses take the full memory access latency.
For each register that is operated on, the latency will depend on which memory region it accesses. 
The total execution time of the instruction will be the sum of the latencies for all register operations. 
If pre or post-indexed addressing mode is used, an extra cycle is added to update the base register, just as regular load/store instructions. 

\subsection{Timing instructions}
Even though the execution time of most timing instructions can be individually be statically determined, they impact the execution time of the program in a very dynamic way.
For example, the execution of \exceptiononexpire\ and \deactivateexception\ themselves only take one cycle, but when the timer expired exception us thrown, the execution time of the whole program is affected. 
In order to reason about the timing effects of the timing instructions, we must understand the implementation effects on the precision of the timing instructions.
It is impossible for any hardware implementation to provide absolute precision of time, as we are limited by the precision of the digital synchronous circuits that discretize the notion of time. 
Although the timing extensions allow the manipulation of nanoseconds in software, with the thread-interleaved pipeline, the basic unit of time for each thread is one thread cycle, or 40 nanoseconds, in PTARM.
As a result, 40 nanoseconds is the shortest interval of time that is observable by each thread.
This can also be explained from the implementation of the pipeline.
Each thread only latches the timer value in the fetch stage, and the timestamp is propagated along the pipeline and associated with the instruction. 
Since there are four threads cycling in a round robin fashion, each thread latches the timer value only once every 4 processor cycles. 
With 100MHz clocking the pipeline in our implementation, 4 processor cycles is  equivalent to 40 nanoseconds. 

When setting deadlines, the execution time of the timing instructions must be taken into account. 
As explained in the implementation, each timing instruction latches the current platform time in the fetch stage.
That timestamp is used throughout instruction execution and associated as the current time.  
For example, the time value in the timestamp obtained from \gettime\ will contain the time before the execution of \gettime.
Since \gettime\ takes 2 thread cycles every execution, 80 nanoseconds will have elapsed when the timestamp is obtained from \gettime.
In the same way, \delayuntil\ latches the current platform time at the fetch stage each thread cycle, and compares with the input timestamp. 
\Delayuntil\ will delay program execution until the latched platform time is \emph{greater than or equal to} the input timestamp value.
So when \delayuntil\ completes its execution, the current platform time will be \emph{at least} 40 nanoseconds greater than the input timestamp specified, to account for the execution of \delayuntil. 

\begin{figure}[h]
  \begin{center}
    \includegraphics[scale=.7]{figs/delay_until_details}
  \end{center}
  \vspace{-10mm}
  \caption{Timing details of get\_time and delay\_until}
  \label{fig:delay_until_details}
\end{figure}

Figure~\ref{fig:delay_until_details} illustrates this effect by showing a timeline of execution for one thread on PTARM. 
The code segment starts executing at time $n$. 
The code only consists of \gettime, \delayuntil, and 2 add instructions used to add an offset to the timestamp obtained by \gettime.
For all 3 cases, the timestamp obtained by \gettime\ would contain the value $n$, even though the instruction after \gettime\ executes at $n+80$.
Taking into account the 2 thread cycles used to add the offset to the timestamp, if the offset is $\leq 160$, then the \delayuntil\ will simply serve as a NOP. 
This is because when \delayuntil\ is executed, it will latch $n+160$ for the platform time, and it will only delay program execution if the input timestamp is $> n+160$.
This is shown in the top case of the figure.
Notice that the instruction after \delayuntil\ executes at time $n+200$, which accounts for the 1 thread cycle it takes to execute \delayuntil.
Assuming \delayuntil\ does delay the program, in the worst-case, the instruction after \delayuntil\ executes $79 ns$ after the input timestamp. 
This can be observed if the offset is set to 161, which will result in the middle timeline shown in figure~\ref{fig:delay_until_details}.  
\Delayuntil\ will first latch the time $n+160$ to compare with the input timestamp of $n+161$. 
Because current platform time is less than input timestamp, \delayuntil\ will delay the execution of the program until the next cycle, when $n+200$ is latched to be checked against the input timestamp. 
At that point, \delayuntil\ will complete its execution, and the next instruction will execute at $n+240$.
This jitter results from the minimum observable time interval of $40 ns$ for each thread, causing \delayuntil\ to have an observable jitter of up to $40 ns$. 
\subsubsection{Timed Loop example revisited}
We give a concrete example of analysis of timing instructions on PTARM by deriving the \emph{offset} from the self compensating timed loop shown in section~\ref{sec:timed_loops}.
This timed loop detects whether the previous loop iteration missed its deadline. 
If it did, then the current iteration will execute a shorter version of the task in attempt to make up for the lost time, as shown in figure~\ref{fig:self_compensating_loop_timing}.  
\begin{figure}[h]
  \vspace{-3mm}
  \begin{center}
    \includegraphics[scale=.7]{figs/self_compensating_loop_timing}
  \end{center}
  \vspace{-10mm}
  \caption{Execution of the self compensating timed loop}
  \label{fig:self_compensating_loop_timing}
\end{figure}

\begin{lstlisting}[float=h, label=lst:timed_loop_compensate_revisit,caption=Timed loops with compensation revisited]
  cdp p13, 8, c2, c0, c0, 0  ; get_time, deadline timestamp stored in [c2, c3]
loop:
  cdp p13, 8, c4, c0, c0, 0  ; get_time, current timestamp stored in [c4, c5]
  subs r5, r5, #80           ; compensate for loop overhead and delay_until 
  sbc  r4, r4, #0            ; 

  subs r3, r3, r5            ; Check if previous iteration deadline is missed
  sbc  r2, r2, r4            ; 

  blmi task_short            ; execute shorter task if previous deadline mess 
  blpl task_normal           ; or else execute normal task 
  
  adds r3, r3, #4000         ; assuming the deadline is 4 us (4000 ns)
  adc r2, r2, #0             ; calculate the deadline timestamp for this iter.
  cdp p13, 4, c2, c2, c3, 0  ; delay_until
   
  b loop
\end{lstlisting}

Listing~\ref{lst:timed_loop_compensate_revisit} shows the source code that is used to construct this timed loop. 
During the miss detection (lines 3 to 8), an additional \textit{offset} is used to compensate for the execution of \delayuntil\ and loop overhead.
Time elapses between the \delayuntil\ of the previous loop iteration (line 15), where the previous deadline timestamp is checked, and the \gettime\ used for miss detection (line 3) in the current iteration.
Without the offset compensation, the loop overhead will cause the miss detection to always detect a missed deadline.
This can be observed from table~\ref{table:timed-loop-compensate-timing}, where we show a sample execution trace of four iterations in this timed loop.  
Figure~\ref{fig:self_compensating_loop_timing} shows the timing behavior of these four iterations, where a missed deadline in the second iteration will cause the third iteration to compensate by executing the shorter version of the task.

In table~\ref{table:timed-loop-compensate-timing}, execution starts at time $t$.
As mentioned before, each thread cycle is $40ns$, which is reflect in the left most column that shows the progression of time.
We also show the thread cycle (TC) count, which starts at $n$ when execution begins.
The execution time of each instruction is according to table~\ref{table:ptarm_instruction_timing}.
In this code segment, we keep track of two timestamps each iteration. 
The \emph{deadline\_timestamp} keeps track of the loop deadlines, and is stored in registers r2 and r3.  
The \emph{current\_timestamp} is updated with \gettime\ in the beginning of each loop iteration to detect if the previous iteration missed its deadline.
It is stored in registers r4 and r5. 
The loop period is set to be $4 \mu s$, which is $4000 ns$ (100 thread cycles).
We add the loop period to the \emph{deadline timestamp} each loop iteration (lines 13 and 14 in the listing).  

\begin{table}
\vspace{-5mm}
\begin{center}
\noindent\makebox[\textwidth]{%
\begin{smalltabular}{ | l | l | l r | }
  \hline
  \textbf{Time} & \textbf{TC} & \textbf{Instruction} & \textbf{Comment}\\ \hline \hline 
  t ns & n &  \textit{cdp p13, 8, c2, c0, c0, 0} & get\_time (deadline: t) \\  \hline
  \multicolumn{4}{|c|}{ -- -- Loop 1st iteration / No deadline miss -- -- } \\ \hline    
  t+80 ns & n+2 &  \textit{cdp p13, 8, c4, c0, c0, 0 } & get\_time, (current: t+80) \\
  t+160 ns & n+4 &  \textit{subs r5, r5, \#80} & (current -= 80)\\
  t+200 ns & n+5 &  \textit{sbc  r2, r2, r4} & (current: t)\\  
  t+240 ns & n+6 &  \textit{subs r3, r3, r5} & compare deadline (t) and current (t) \\
  t+280 ns & n+7 &  \textit{sbc  r2, r2, r4} & result is 0, clear cc[``n''] \\
  t+320 ns & n+8 &  \textit{blmi task\_short} & nop since cc[``n''] == 0\\
  t+360 ns & n+9 &   \textit{blpl task\_normal} & branch since cc[``n"] == 0\\  
    - ns & - &  \ldots & executing task\_normal \\  
  t+3800 ns & n+95 & \textit{adds r3, r3, \#4000} &  (deadline += 4000) \\
  t+3840 ns & n+96 & \textit{adc r2, r2, \#0} &  (deadline: t+4000) \\ 
  t+3880 ns & n+97 &  \textit{cdp p13, 4, c2, c2, c3, 0} & delay\_until, input timestamp is t+4000\\
  - ns & - & \ldots & delay\_until for 3 thread cycles\\  
  t+4040 ns & n+101 &  \textit{b loop} & jump back to loop \\ \hline
  \multicolumn{4}{|c|}{ -- -- Loop 2nd iteration / Deadline miss -- -- } \\ \hline    
  t+4080 ns & n+102 &  \textit{cdp p13, 8, c4, c0, c0, 0 } & get\_time, (current: t+4080)\\
  t+4160 ns & n+104 &  \textit{subs r5, r5, \#80} & (current -= 80)\\
  t+4200 ns & n+105 &  \textit{sbc  r2, r2, r4} & (current: t+4000)\\  
  t+4240 ns & n+106 &  \textit{subs r3, r3, r5} & compare deadline (t+4000) and current (t+4000)\\
  t+4280 ns & n+107 &  \textit{sbc  r2, r2, r4} & result is 0, clear cc[``n''] \\
  t+4320 ns & n+108 &  \textit{blmi task\_short} & nop since cc[``n''] == 0\\
  t+4360 ns & n+109 &   \textit{blpl task\_normal} & branch since cc[``n"] == 0\\  
  - ns & - &  \ldots & code for task\_normal \\  
  t+7960 ns & n+199 & \textit{adds r3, r3, \#4000} & (deadline += 4000) \\
  t+8000 ns & n+200 & \textit{adc r2, r2, \#0} & (deadline: t+8000) \\ 
  t+8040 ns & n+201 &  \textit{cdp p13, 4, c2, c2, c3, 0} & delay\_until, *no delay*\\
  t+8080 ns & n+202 &  \textit{b loop} & jump back to loop \\ \hline
  \multicolumn{4}{|c|}{ -- -- Loop 3rd iteration / Compensate with shorter task -- -- } \\ \hline    
  t+8120 ns & n+203 &  \textit{cdp p13, 8, c4, c0, c0, 0 } & get\_time, (current: t+8120)\\
  t+8200 ns & n+205 &  \textit{subs r3, r3, r5} & (current -= 80)\\
  t+8240 ns & n+206 &  \textit{sbc  r2, r2, r4} & (current: t+8040) \\
  t+8280 ns & n+207 &  \textit{subs r3, r3, r5} & compare deadline (t+8000) and current (t+8040)\\
  t+8320 ns & n+208 &  \textit{sbc  r2, r2, r4} & result is -40, set cc[``n''] \\
  t+8360 ns & n+209 &  \textit{blmi task\_short} & branch since cc[``n"] == 1 \\
  - ns & - &  \ldots & code for task\_short \\  
  t+10280 ns & n+257 &   \textit{blpl task\_normal} & nop since cc[``n''] == 1\\    
  t+10320 ns & n+258 & \textit{adds r3, r3, \#4000} & (deadline += 4000) \\
  t+10360 ns & n+259 & \textit{adc r2, r2, \#0} & (deadline: t+12000) \\ 
  t+10400 ns & n+260 &  \textit{cdp p13, 4, c2, c2, c3, 0} & delay\_until\\
  - ns & - &  \ldots & delay until time is t+12000 \\  
  t+12040 ns & n+301 &  \textit{b loop} & jump back to loop \\ \hline
   \multicolumn{4}{|c|}{ -- -- Loop 4th iteration / Execute normal task -- -- } \\ \hline    
  t+12080 ns & n+302 &  \textit{cdp p13, 8, c4, c0, c0, 0 } & get\_time, (current: t+12080)\\
  t+12160 ns & n+304 &  \textit{subs r3, r3, r5} & (current -= 80)\\
  t+12200 ns & n+305 &  \textit{sbc  r2, r2, r4} & (current: t+12000) \\
  t+12240 ns & n+306 &  \textit{subs r3, r3, r5} & compare deadline (t+12000) and current (t+12000)\\
  t+12280 ns & n+307 &  \textit{sbc  r2, r2, r4} & result is 0, clear cc[``n''] \\
  \hline 
\end{smalltabular}}
\end{center}
\vspace{-5mm}
\caption{Self compensating timed loop timing details (TC = thread cycles)}
\label{table:timed-loop-compensate-timing}
\end{table}


\newcommand{\currentt}{\emph{current\_timestamp}}
\newcommand{\deadlinet}{\emph{deadline\_timestamp}}

The need for the \emph{offset} can be observed at the beginning of the second loop iteration.
At time $t+4080ns$, \gettime\ is called to initiate the miss detection sequence.
The previous \deadlinet\ is $t+4000$, which was met in the first iteration.  
However, \gettime\ updates the \currentt\ to $t+4080$, because the execution of \delayuntil\ and \emph{b loop} took 2 thread cycles combined. 
Thus, our miss detection accounts for this by subtracting the 2 thread cycles ($80ns$) overhead from \currentt\ before comparing it with \deadlinet.
In general, the overhead that needs to be account for is the time elapsed between the deadline checking \delayuntil\ instruction and the miss detection \gettime\ instruction.
Intuitively this is because we want to check whether the previous \delayuntil\ executed before the previous \deadlinet, so the \emph{offset} is used to calculate the time of execution of the previous \delayuntil.  

\begin{figure}[h]
  \vspace{-3mm}
  \begin{center}
    \includegraphics[scale=.7]{figs/setup_loop_timing}
  \end{center}
  \vspace{-10mm}
  \caption{Jitter caused by initial timed loop setup}
  \label{fig:setup_look_timing}
\end{figure}

In our code, the loop setup
When the first loop iteration executes, the \currentt\ will first be checked against the \deadlinet\ that was setup before the loop began execution.
The compensation for the loop offset in this case also helped compensate for the initial loop setup time.
Without it, the \currentt\ would be greater than \deadlinet\ from the execution time of the setup \gettime instruction, and cause a false missed deadline detection.
Because it takes 2 thread cycles ($80ns$) to execute \gettime, the value the \textbf{offset} at a minimum must be 2 thread cycles so the first iteration will not falsely detect a missed deadline.  
As execution continues, the first iteration executes the normal task, and the \delayuntil\ is executed before the first deadline of $t+4000ns$.




\subsection{Exceptions}
The exception\_on\_expire instruction is used to handle missed deadlines immediately by triggering a hardware exception. 
Once the deadline is setup using this instruction, the PTARM hardware will throw a hardware exception immediately when the timer value passes the deadline.
The PTARM architecture provides one slot for each thread to register a deadline that is checked in hardware.
Each processor cycle the timer checks against these registered deadlines in hardware, and throws a hardware exception for the corresponding thread when the timer exceeds the deadline value. 
The deactivate\_exception is used to un-register the deadline value before the deadline is reached. 
This way no exception will be thrown.

When a hardware exception is thrown, the architecture modifies the program counter (PC) according to an exception vector table, and stores the current PC to the link register. 
The current processor state register (CPSR) is saved, and normal interrupts are disabled. 
In the case of a missed deadline, since it is a custom exception, we added an entry to the exception vector table to point to a custom timer expire exception handler code. 
The entry added changes the PC to the address 0x0000001c, which contains an instruction that branches to our setup code for the timer expire handler. 
The setup code is short, and mainly is used to jump to a user specified timer expire handler code.
As shown in listing~\ref{timer-expire-handler}, we reserve one word (\textit{\_timer\_handler\_loc}) for the user to write the address of a timer expire handler function. 
Assuming that the user already has written an address to \textit{\_timer\_handler\_loc}, then our setup code loads that address into the PC, and executes the user specified function. 
When the function returns to this setup code, we re-enable interrupts, and returns to the location when the exception occurred.
It is important to know the location of \textit{\_timer\_handler\_loc} because the user must write the address of their timer handler to that location.
Our compiled C run time code puts the \textit{\_timer\_handler\_loc} at address 0x00000098.
If the C run time code is recompiled, then each program must update where the timer handler address is written to to the new address of \textit{\_timer\_handler\_loc}.
Listing~\ref{exception-sample} shows an example usage of exception\_on\_expire and deactivate\_exception.    
\begin{lstlisting}[float, label=timer-expire-handler,caption=The custom timer expire setup code]
.text
.global _tmr_exp_setup;
_tmr_exp_setup:
    push  {r0, lr}
    ldr   r0, _timer_handler_loc 
    mov   lr, pc
    mov   pc, r0
    mrs   r0, cpsr
    bic   r0, r0, #0x80
    msr   cpsr, r0
    pop   {r0, pc}

_timer_handler_loc: .word  0x00000000;
\end{lstlisting}

\begin{lstlisting}[float, label=exception-sample,caption=Sample assembly code of exception\_on\_expire and deactivate\_exception ]
  mov r3, #0x98
  add r4, pc, #32            ;r4 = addr of delay_handler
  str r4, [r3]               ;register delay_handler
  mov r0, #80
  mov r1, #0
  cdp p13, 8, c2, c0, c1, 0  ;set_time
  cdp p13, 2, c2, c2, c3, 0  ;exception_on_expire
  add r5, r6, r6
  add r7, r5, r6
  cdp p13, 5, c8, c2, c3, 0  ;deactivate_exception
  b end
delay_handler:
  mov pc, lr
\end{lstlisting}

The first three instructions in the sample code registers ``delay\_handler'' to be the timer expire handler.
The code for delay\_handler currently does not do anything, and simply returns to its caller, we use it for illustrative purposes. 
We use set\_time to obtain the desired deadline, which is stored in r2 and r3.
We then use those to register an exception to trigger when the PTARM timer reaches that time. 
The deactivate\_exception instruction doesn't take in any operands, but simply disarms the timer from throwing a hardware exception. 
In the sample code above, we purposely set the deadline so a timer expire exception will always occur. 
Table~\ref{exception-expire-timing} shows the timing details and instruction execution sequence of the above code.

\begin{table}
\begin{center}
\noindent\makebox[\textwidth]{%
\begin{smalltabular}{ | c | c | l | l r | }
  \hline                        
  Time & TC & address & Inst & Comment\\ \hline
  0 ns & 0 & 0x40000000 & \textit{mov r3, \#0x98} & get the timer handler address \\  
  40 ns & 1 & 0x40000004 & \textit{add r4, pc, \#32} & get delay\_handler address \\
  80 ns & 2 & 0x40000008 & \textit{str r4, [r3]} & register delay\_handler as timer expire handler\\
  120 ns & 3 & 0x400000C & \textit{mov r0, \#80} & setup offset low 32 bits \\
  160 ns & 4 & 0x40000010 & \textit{mov r1, \#0} & setup offset high 32 bits \\
  200 ns & 5 & 0x40000014 & \textit{cdp p13, 8, c2, c0, c1, 0} & set\_time instruction (deadline 360) \\
  240 ns & 6 & & & \\
  280 ns & 7 & 0x40000018 & \textit{cdp p13, 2, c2, c2, c3, 0} & exception\_on\_expire instruction (deadline 360) \\
  320 ns & 8 & 0x4000001C & \textit{add r5, r6, r6} & code block\\
  360 ns & 9 & 0x40000020 & **throw exception** & timer expired, hardware exception thrown\\  
  400 ns & 10 & 0x1C & \textit{b \_tmr\_exp\_setup } & branch to setup code \\  
  440 ns & 11 & 0x78 & \textit{push \{r0, lr\}} & store registers on stack\\
  480 ns & 12 & & & \\
  520 ns & 13 & 0x7C & \textit{ldr   r0, \_timer\_handler\_loc} & load the address of timer handler \\
  560 ns & 14 & 0x80 & \textit{mov   lr, pc} & store return address after timer handler \\
  600 ns & 15 & 0x84 & \textit{mov   pc, r0} & jump to registered handler (delay\_handler) \\
  640 ns & 16 & 0x4000002C & \textit{mov   pc, lr} & delay\_handler code, jump back to setup code\\  
  680 ns & 17 & 0x88 & \textit{mrs   r0, cpsr} & load the CPSR\\
  720 ns & 18 & 0x8C & \textit{bic   r0, r0, \#0x80} &  enable interrupts\\
  760 ns & 19 & 0x90 & \textit{msr   cpsr, r0} &  store the CPSR\\
  800 ns & 20 & 0x94 & \textit{pop   \{r0, pc\}} & return from interrupt\\
  840 ns & 21 & & & \\    
  880 ns & 22 & 0x40000020 & \textit{add r7, r5, r6} & code block\\
  920 ns & 23 & 0x40000024 & \textit{cdp p13, 3, c2, c0, c1, 0} & deactivate\_exception (doesn nothing) \\
  960 ns & 24 & 0x40000028 & \textit{b end} & instruction after return from delay\_handler\\
  \hline 
\end{smalltabular}}
\end{center}
\caption{Exception\_on\_expire sample code timing details}
\label{exception-expire-timing}
\end{table}

Both exception\_on\_expire and deactivate\_exception take only 1 cycle to complete.
The execution sequence shown in table~\ref{exception-expire-timing} jumps back and forth between the main code, timer expired setup code, and the actual delay\_handler. 
In the case that the deadline was set to be longer, the deactivate\_exception instruction would have been executed before the timer expired, and a hardware exception would not have been thrown.









