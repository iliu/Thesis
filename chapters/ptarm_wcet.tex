\label{subsec:precision_timing_inst_ptarm}

%mention that a full scaled worst case execution time analysis is beyond the scope of this thesis.
%but do a brief summary on how it's done, cite wilhelm's group's research summary
Worst-case execution time (WCET) analysis is a combined analysis of the control paths that the software might exhibit, and the time it takes to execute those paths on the underlying architecture. 
A plethora of research has been done on the software analysis of control paths. 
Wilhelm et al.~\cite{wilhelm-survey-paper} presented a survey of tools and techniques available for worst-case path enumeration and loop analysis etc.
However, the precision of the WCET analysis of those techniques ultimately depend on the underlying architecture implementation~\cite{Heckmann2003processor}.
Architectures that exhibit wildly unpredictable execution times will result in overly conservative WCET analysis, even if the software structure is simple. 
Designed as a predictable architecture, the instructions of PTARM all exhibit deterministic timing behaviors, allowing a more precise WCET as software analysis progresses. 
In section~\ref{sec:ptarm_instructions} we discussed the implementation of each instruction and explained the execution time of each instruction type.
Table~\ref{table:ptarm_instruction_timing} summarizes the execution time each instruction takes in terms of \emph{thread cycles}. 

A \emph{Thread cycle} is the unit used to represent execution time on each thread.  
Timing analysis can be done separately for each hardware thread running on PTARM because the threads are temporally isolated; the execution time of each thread cannot be affected by other threads.
The thread-interleaved pipeline switches thread contexts every processor cycle in a predictable round robin fashion. 
Thus, each thread is fetched and executed in the pipeline every $N$ processor cycles, $N$ being the number of threads in the pipeline.
One \emph{Thread cycle} represents each time the thread enters in the pipeline, which is the thread's perceived notion of cycles.
The execution frequency of each thread ($F_{thread}$) is $F_{thread} = F_{processor}/N$, so each \emph{thread cycle} is $1/F_{thread}$ long. 
For example, our PTARM core is clocked at 100$MHz$ ($F_{processor} = 100 * 10^6$) and has 4 threads ($N=4$) , so each thread cycle is $1/((100 * 10^6)/4) = 40 * 10^{-9}$ secs, or 40 nanoseconds long.
The length of the \emph{thread cycle} will not change because of the predictable thread-switching policy, making it a reliable unit of measurement for execution time.   

\begin{table}
\noindent\makebox[\textwidth]{%
\begin{smalltabular}{ | p{4cm} | c || p{9cm} | }
  \hline                        
  \textbf{Instruction Type} & \textbf{Thread Cycles} & \textbf{Notes}    \\ \hline
  Data Processing  & 1 & \multirow{2}{9cm}{ ${\ast}$: an additional cycle is added if pre or post-index offset is used in the address mode.}  \\ \cline{1-2} 
  Branch  & 1 &  \\ \cline{1-2}
  Load/Store (SPM)  & 1$^{\ast}$ &  \multirow{4}{9cm}{ ${\phi}$: the dram latency is 3 or 4 thread cycles depending on the alignment of the pipeline and the dram controller backend, as described in section~\ref{sec:ptarm_dram_integration}. For conservative estimates, 4 thread cycles is used. } \\ \cline{1-2} 
  Load/Store (BootROM)  & 1$^{\ast}$ & \\ \cline{1-2}  
  Load/Store (DRAM) & 4$^{\phi \dagger \ast}$ & \\ \cline{1-2}
  Load/Store Multiple & $N_{reg} \times L_{mem}$ $^{\Delta \ast}$ &  \\  \cline{1-2}
  Software Interrupt (SWI) & 1 & \multirow{4}{9cm}{ ${\dagger}$: a single store buffer is implemented, as described in section~\ref{sec:ptarm_dram_store_buffer}, so non-consecutive stores take only 1 thread cycle, while consecutive stores require the full dram latency of 3 or 4.}\\    \cline{1-2}
  get\_time  & 2 &  \\ \cline{1-2}
  delay\_until  & 1 $^{\sigma}$ & \\    \cline{1-2}
  exception\_on\_expire  & 1  &    \\   \cline{1-2}
  deactivate\_exception & 1 &  \multirow{2}{9cm}{ ${\Delta}$: $N_{reg}$ is number of registers specified in the instruction. $L_{mem}$ is the latency of the memory region operated on.} \\  \cline{1-2}
  \multicolumn{2}{|c|}{} &  \\ \cline{1-2}
  \multicolumn{2}{|c|}{} &   \multirow{2}{9cm}{${\sigma}$: denotes the minimum execution time. Actual execution time depends on the system clock and the specified deadline.} \\ \cline{1-2}
  \multicolumn{2}{|c|}{} &\\ \hline
\end{smalltabular}}
\caption{Timing properties of PTARM instructions}
\label{table:ptarm_instruction_timing}
\end{table}

\subsection{Memory instructions}
Data-processing instructions typically have straightforward execution times on most architectures.
In our case, branch instructions also have very a predictable and straightforward execution, as the branch penalty is completely hidden within the thread interleaving.
The memory instructions in our architecture however can have several different latencies depending on addressing mode or region of access, as listed in table~\ref{table:ptarm_instruction_timing}.
For memory instructions that use pre or post-indexed addressing mode to update the base register, an additional cycle latency is needed to write back to the base register, as described in the instruction implementation in section~\ref{sec:ptarm_instruction_ldstr}.
But the addressing mode of load/store instructions can be determined statically, as it is part of the instruction encoding, so it does not affect the complexity or precision of execution time analysis. 

For load instructions, the exposed memory hierarchy allows us to clearly label and identify access latencies to different memory regions.
In execution time analysis, value analysis attempts to determine what memory addresses are accessed.     
In systems that use caches and hide the memory hierarchy, additional modeling of the cache state is still needed after the value analysis determines the potential memory addresses that are accessed.
However, with an exposed memory hierarchy with scratchpads, the precision of execution time analysis depends only on the value analysis. 
As soon as the memory address is determined, we can give a associate a precise memory access latency.
The worst-case DRAM latency of 4 thread cycles is derived from~\cite{ReinekeLiuPatelKimLee11_PRETDRAMControllerBankPrivatizationForPredictability}. 
This allows not only for a simpler timing analysis, but also a more accurate analysis.

For store instructions, the single store buffer as described in section~\ref{sec:ptarm_dram_store_buffer} can possibly hide the latency of store instructions if no other memory access instruction comes after it. 
Otherwise the store instruction will take the full memory access latency, depending on the memory access region of the store.
Timing analysis tools can mostly account for the store buffer by statically checking a few instructions ahead of the store instruction to see if there are memory accessing instructions.
The window of instructions that need to be checked is very narrow, so it should not really complicate the analysis.
If this is not possible, then conservatively, the full memory access latency should be used for each store, depending on the access region.

The execution time of load/store multiple instructions depend on the number of registers operated on, and the memory region it accesses. 
Because the register list is statically encoded in the instruction, the number of registers operated on can easily be statically determined.
Store multiple instructions do not benefit from the store buffer, so all accesses take the full memory access latency.
For each register that is operated on, the latency will depend on which memory region it accesses. 
The total execution time of the instruction will be the sum of the latencies for all register operations. 
If pre or post-indexed addressing mode is used, an extra cycle is added to update the base register, just as regular load/store instructions. 

\subsection{Timing instructions}
The execution time of most timing instructions can be statically determined. 
Their effect on the execution time of the program though could dynamically change however, depending on the deadlines used and the current platform time.
The \gettime\ instruction takes 2 thread cycles every execution, as it simply obtains the current platform time.
Although \exceptiononexpire\ and \deactivateexception\ can affect the execution time of the program by triggering timer expired exceptions, the execution of the instructions themselves only take 1 thread cycle.
The single thread cycle merely sets up or deactivates the correct hardware mechanisms for checking the timer expired exceptions.  
The execution time of \delayuntil\ will depend on the current platform time and its input deadline timestamp.
This by design the semantics of \delayuntil. 
At a minimum, \delayuntil\ will still take one thread cycle to execute, even if the current platform time has passed the input timestamp.

In order to reason about the timing effects of the timing instructions, we must understand the implementation effects on the precision of the timing instructions.
It is impossible for any hardware implementation to provide absolute precision of time, as we are limited by the precision of the digital synchronous circuits that discretize the notion of time. 
In our timing extensions, we allow the users to manipulate a 64 bit nanosecond representation of time in software. 
However, because of the thread-interleaved pipeline, our basic unit of time is limited to one thread cycle, which is 40 nanoseconds in our implementation.
This has a subtle effect on the minimum precision that the timing instructions can support.
   

 
\subsubsection{Timed Loop example revisited}
\begin{lstlisting}[float=h, label=lst:timed_loop_compensate,caption=Timed loops with compensation ]
  cdp p13, 8, c2, c0, c0, 0  ; get_time, deadline timestamp stored in [c2, c3]
loop:
  cdp p13, 8, c4, c0, c0, 0  ; get_time, current timestamp stored in [c4, c5]
  subs r5, r5, #<offset>     ; <offset> is implementation dependent and used to 
  sbc  r4, r4, #0            ; account for loop overhead and miss detection

  subs r3, r3, r5            ; Check if previous iteration deadline is missed
  sbc  r2, r2, r4            ; 

  blmi task_short            ; execute shorter task if previous deadline mess 
  blpl task_normal           ; or else execute normal task 
  
  adds r3, r3, #0xDEAD       ; assuming the deadline is #0xDEAD
  adc r2, r2, #0             ; calculate the deadline timestamp for this iter.
  cdp p13, 4, c2, c2, c3, 0  ; delay_until
   
  b loop
\end{lstlisting}


\subsection{Exceptions}
The exception\_on\_expire instruction is used to handle missed deadlines immediately by triggering a hardware exception. 
Once the deadline is setup using this instruction, the PTARM hardware will throw a hardware exception immediately when the timer value passes the deadline.
The PTARM architecture provides one slot for each thread to register a deadline that is checked in hardware.
Each processor cycle the timer checks against these registered deadlines in hardware, and throws a hardware exception for the corresponding thread when the timer exceeds the deadline value. 
The deactivate\_exception is used to un-register the deadline value before the deadline is reached. 
This way no exception will be thrown.

When a hardware exception is thrown, the architecture modifies the program counter (PC) according to an exception vector table, and stores the current PC to the link register. 
The current processor state register (CPSR) is saved, and normal interrupts are disabled. 
In the case of a missed deadline, since it is a custom exception, we added an entry to the exception vector table to point to a custom timer expire exception handler code. 
The entry added changes the PC to the address 0x0000001c, which contains an instruction that branches to our setup code for the timer expire handler. 
The setup code is short, and mainly is used to jump to a user specified timer expire handler code.
As shown in listing~\ref{timer-expire-handler}, we reserve one word (\textit{\_timer\_handler\_loc}) for the user to write the address of a timer expire handler function. 
Assuming that the user already has written an address to \textit{\_timer\_handler\_loc}, then our setup code loads that address into the PC, and executes the user specified function. 
When the function returns to this setup code, we re-enable interrupts, and returns to the location when the exception occurred.
It is important to know the location of \textit{\_timer\_handler\_loc} because the user must write the address of their timer handler to that location.
Our compiled C run time code puts the \textit{\_timer\_handler\_loc} at address 0x00000098.
If the C run time code is recompiled, then each program must update where the timer handler address is written to to the new address of \textit{\_timer\_handler\_loc}.
Listing~\ref{exception-sample} shows an example usage of exception\_on\_expire and deactivate\_exception.    
\begin{lstlisting}[label=timer-expire-handler,caption=The custom timer expire setup code]
.text
.global _tmr_exp_setup;
_tmr_exp_setup:
    push  {r0, lr}
    ldr   r0, _timer_handler_loc 
    mov   lr, pc
    mov   pc, r0
    mrs   r0, cpsr
    bic   r0, r0, #0x80
    msr   cpsr, r0
    pop   {r0, pc}

_timer_handler_loc: .word  0x00000000;
\end{lstlisting}

\begin{lstlisting}[label=exception-sample,caption=Sample assembly code of exception\_on\_expire and deactivate\_exception ]
  mov r3, #0x98
  add r4, pc, #32            ;r4 = addr of delay_handler
  str r4, [r3]               ;register delay_handler
  mov r0, #80
  mov r1, #0
  cdp p13, 8, c2, c0, c1, 0  ;set_time
  cdp p13, 2, c2, c2, c3, 0  ;exception_on_expire
  add r5, r6, r6
  add r7, r5, r6
  cdp p13, 5, c8, c2, c3, 0  ;deactivate_exception
  b end
delay_handler:
  mov pc, lr
\end{lstlisting}

The first three instructions in the sample code registers ``delay\_handler'' to be the timer expire handler.
The code for delay\_handler currently does not do anything, and simply returns to its caller, we use it for illustrative purposes. 
We use set\_time to obtain the desired deadline, which is stored in r2 and r3.
We then use those to register an exception to trigger when the PTARM timer reaches that time. 
The deactivate\_exception instruction doesn't take in any operands, but simply disarms the timer from throwing a hardware exception. 
In the sample code above, we purposely set the deadline so a timer expire exception will always occur. 
Table~\ref{exception-expire-timing} shows the timing details and instruction execution sequence of the above code.

\begin{table}
\begin{center}
\noindent\makebox[\textwidth]{%
\begin{tabular}{ | c | c | l | l r | }
  \hline                        
  Time & TC & address & Inst & Comment\\ \hline
  0 ns & 0 & 0x40000000 & \textit{mov r3, \#0x98} & get the timer handler address \\  
  40 ns & 1 & 0x40000004 & \textit{add r4, pc, \#32} & get delay\_handler address \\
  80 ns & 2 & 0x40000008 & \textit{str r4, [r3]} & register delay\_handler as timer expire handler\\
  120 ns & 3 & 0x400000C & \textit{mov r0, \#80} & setup offset low 32 bits \\
  160 ns & 4 & 0x40000010 & \textit{mov r1, \#0} & setup offset high 32 bits \\
  200 ns & 5 & 0x40000014 & \textit{cdp p13, 8, c2, c0, c1, 0} & set\_time instruction (deadline 360) \\
  240 ns & 6 & & & \\
  280 ns & 7 & 0x40000018 & \textit{cdp p13, 2, c2, c2, c3, 0} & exception\_on\_expire instruction (deadline 360) \\
  320 ns & 8 & 0x4000001C & \textit{add r5, r6, r6} & code block\\
  360 ns & 9 & 0x40000020 & **throw exception** & timer expired, hardware exception thrown\\  
  400 ns & 10 & 0x1C & \textit{b \_tmr\_exp\_setup } & branch to setup code \\  
  440 ns & 11 & 0x78 & \textit{push \{r0, lr\}} & store registers on stack\\
  480 ns & 12 & & & \\
  520 ns & 13 & 0x7C & \textit{ldr   r0, \_timer\_handler\_loc} & load the address of timer handler \\
  560 ns & 14 & 0x80 & \textit{mov   lr, pc} & store return address after timer handler \\
  600 ns & 15 & 0x84 & \textit{mov   pc, r0} & jump to registered handler (delay\_handler) \\
  640 ns & 16 & 0x4000002C & \textit{mov   pc, lr} & delay\_handler code, jump back to setup code\\  
  680 ns & 17 & 0x88 & \textit{mrs   r0, cpsr} & load the CPSR\\
  720 ns & 18 & 0x8C & \textit{bic   r0, r0, \#0x80} &  enable interrupts\\
  760 ns & 19 & 0x90 & \textit{msr   cpsr, r0} &  store the CPSR\\
  800 ns & 20 & 0x94 & \textit{pop   \{r0, pc\}} & return from interrupt\\
  840 ns & 21 & & & \\    
  880 ns & 22 & 0x40000020 & \textit{add r7, r5, r6} & code block\\
  920 ns & 23 & 0x40000024 & \textit{cdp p13, 3, c2, c0, c1, 0} & deactivate\_exception (doesn nothing) \\
  960 ns & 24 & 0x40000028 & \textit{b end} & instruction after return from delay\_handler\\
  \hline 
\end{tabular}}
\end{center}
\caption{Exception\_on\_expire sample code timing details}
\label{exception-expire-timing}
\end{table}

Both exception\_on\_expire and deactivate\_exception take only 1 cycle to complete.
The execution sequence shown in table~\ref{exception-expire-timing} jumps back and forth between the main code, timer expired setup code, and the actual delay\_handler. 
In the case that the deadline was set to be longer, the deactivate\_exception instruction would have been executed before the timer expired, and a hardware exception would not have been thrown.









