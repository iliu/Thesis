We are certainly not the first or only one to tackle the unpredictability of computer architecture designs.
In this chapter we survey the abundance of related research to our goal of predictable architectures.
Timing analysis techniques, compiler techniques and architectural techniques all play a role in tackling the unpredictability of computer architectures.
However, we limit the scope of this survey to a few compiler techniques with the main focus on architectural techniques, as that is the focus of this thesis. 

Adding temporal semantics to programming languages has been the focus of many research proposals, but to the best of our knowledge, we believe this is the first attempt to introduce temporal semantics down to the ISA level. 

%FIXME: Compare to XMOS style handling of interrupts?

\section{Pipeline-Focused Techniques}
\subsection{Static Branch Predictors}
\label{sec:RTBranch}
Dynamic branch predictors cause timing anomalies~\cite{Engblom2003dynbranch}, and are difficult to model is because of the \textit{aliasing} of branch points.   
\textit{Aliasing} occurs when two different branches occupy the same branch predictor slot, and cause interference.
Burguiere et al.~\cite{Burguiere2005staticbranchpredict} made a case for \textit{static branch prediction} to be used for real-time systems.
This could be done in several ways. 
The simplest form can predict all branches taken or not taken. 
Improvements can include the \textit{Backward Taken, Forward Not taken} scheme, to improve performance for loops and if statements.
This uses the observation that for loop control branches, most all backwards branches are taken to return to the loop, and only at the end of the loop are forward branches taken. 
With architecture support for static branch predictions, compilers can analyze code patterns (loops, if-then-else, if then) and insert instruction set constructs to denote the static prediction of each branch.  
The underlying architecture will use that for its prediction, instead of relying on a dynamic hardware unit.
This removes \textit{aliasing} and gives better estimated worst case branch mispredicts. 

Bodin et al.~\cite{Bodin2005staticbranch} uses this idea of software static branch prediction to improve WCET of programs.
Intuitively, they aim to remove all branch mispredict penalties from the worst-case path to improve the WCET. 
They propose an algorithm that iterates through the control flow graph to find the worst-case execution path (WCEP). 
Initially, the algorithm finds the worst case path assuming all the branches are mispredicted. 
Then, the algorithm assigns the static branch prediction of all branches on the WCEP to be taken.
The algorithm then iterates again to find the new WCEP until two iterations yield the same WCEP.
Since the algorithm never reassigns assigned branches, it always converges but is not optimal.
The presence of caches can easily effect the WCEP, and each branch prediction reassigned could modify the cache state.
However, the experiments assumed all code and data fit into the caches, thus the effect of caches were not factored into by the algorithm.

\subsection{Superscalar Pipelines}
\label{sec:RTSuperscale}
Superscalar pipelines issue multiple instructions at a time to exploit instruction-level parallelism (ILP). 
In order to keep the pipeline filled, superscalar pipelines typically employ more aggressive techniques to fully utilize the ILP. 
As a result, attempting to model all advanced techniques often leads to either very pessimistic results, or almost infeasible complex models.

Rochange et al.~\cite{Rochange2005superscalar} propose to use instruction pre-scheduling to ease the difficulties of analysis of superscalar pipelines.  
The concept is similar to resetting the pipeline state before each basic block execution. 
This is done by postponing the scheduling of instructions from the next basic block until the instructions from the previous basic block are completed.
If it is possible to remove all timing interference across basic blocks, then the resources needed to model the pipeline could be significantly reduced, as each basic block will start with a consistent initial state.
However, the results assume the absence of caches, which could easily effect execution across basic blocks.  
Furthermore, depending on how many instructions can be in flight at one time, waiting for the pipeline state to be flushed could induce large penalties for programs with a lot of control flow transfer and small basic blocks. 

% \paragraph{A time-predictable execution mode for superscalar pipelines with instruction prescheduling ~\cite{Rochange2005superscalar}}
% \begin{itemize}
% \item What is the background of this work? What is the motivation?\\
% 
% \item What is the main goal?\\
%   They want to reconcile high performance with time
%   predictability. Mainly, they are making out of order superscalar
%   pipelines fit WCET estimation techniques. 
% \item What did they do to achieve this goal?\\
%   They control instruction flow to remove dependence between basic
%   blocks so that any WCET estimation tool would only need to measure
%   or estimate a smaller segment of code.  
% \item How do they evaluate their approach? Does it achieve the goal?
%   Do they compare it with other work?\\
%   They showed a performance comparison of slow down compared to
%   regular out of order superscalar and also against in order scalar
%   pipeline to show performance improvement. 
% \item What other work is listed as future work?\\
% \end{itemize}
% Additional questions:
% \begin{itemize}
%   \item What are the limitations/assumptions of this work?\\
%     They ignore peripheral components (cache memories, TLBs) as well
%     as external events (interrupts) and interactions with the
%     operating system (process scheduling, virtual memory, etc).
%   \item Which parts of a system/design process are modified by this
%     work? (e.g. hardware (which feature?), WCET analysis, scheduling,
%     compiler, programming language, \ldots)\\
% \end{itemize}


% \paragraph{Predictable Out-of-order Execution Using Virtual Traces ~\cite{whitham:08:predOOOwithVirtualTraces}}
% \begin{itemize}
% \item What is the background of this work? What is the motivation?\\
%   The motivation is to improve WCET of complex processors,
%   specifically for out-of-order superscalar pipelines. 
% \item What is the main goal?\\
%   The main goal of this paper is 3 fold. 1) minimizing the pessimism
%   introduced in WCET analysis. 2) increasing CPU throughput that can
%   be guaranteed. 3) minimize CPU modeling cost. 
% \item What did they do to achieve this goal?\\
%   They introduced a VTC (virtual trace controller) to control the
%   progress of the pipeline. They argue that this controller can be
%   used for a CPU of arbitrary complexity. The VTC operates CPU
%   programs as a collection of traces. Traces are paths through the
%   program. Essentially traces are formed by statically predicting
%   branches, in the context of this paper they predict the branches
%   towards the worst case execution path. This way the pipeline
%   optimizations (out of order execution etc) can optimize the worst
%   case path. This achieves goal 2, which is increased the guaranteed
%   throughput. The traces are formed via static branch predictions, and
%   the VTC contains a VTR (virtual trace register) which stores the
%   branch predictions. The pipeline state is reset between traces, so
%   the WCET analysis can be limited to within traces. The side exits
%   are determined by branch mispredicts. Now WCET within each trace and
%   side exit can be measured, and it will be the same execution time,
%   thus no CPU modeling cost is needed. This achieves goals 1 and 3. 
% \item How do they evaluate their approach? Does it achieve the goal?
%   They use the Malardalen WCET benchmark suite, but assume that
%   benchmark programs are single-path programs. They assume using IPET
%   or methods can find the WCET. They use the benchmark programs and
%   run the program on an idealized in order machine to compare the
%   results with their machine. They compare the speed up /slow down and
%   use it to analyze the issues with their approach. 
% \item What other work is listed as future work?\\
% \end{itemize}
% Additional questions:
% \begin{itemize}
%   \item What are the limitations/assumptions of this work?\\
%     They assume WCEP is easily obtained. It also depends on how many
%     traces are formed, and how effective the traces are formed. They
%     assume scratchpads in this work.
%   \item Which parts of a system/design process are modified by this
%     work? (e.g. hardware (which feature?), WCET analysis, scheduling,
%     compiler, programming language, \ldots)\\
%     They need a compiler to compile code into traces and form
%     traces. The hardware is modified with a VTC to control the traces
%     and stall the pipeline between traces. 
% \end{itemize}


Whitham et al. \cite{whitham:08:predOOOwithVirtualTraces} combined the techniques of \emph{instruction pre-scheduling} and \emph{static branch predictions} and propose modifications to an out-of-order superscalar pipeline to provide predictability for single thread execution.  
Instead of basic blocks, the superscalar pipeline pre-schedules instructions across \emph{virtual traces}\cite{Whitham2008formvirtualtraces}. 
\emph{Virtual traces} are program paths with static branch predictions inserted. 
These are usually formed by predicting along the WCEP, similar to the algorithm Bodin et al.~\cite{Bodin2005staticbranch} introduced. 
Each virtual trace can contain a fixed number of branches. 
A VTC (virtual trace controller) is introduced to control the progress of the pipeline.
The VTC contains a VTR (virtual trace register) which stores the branch predictions, and the pipeline state is reset between traces so the WCET analysis can be limited to within traces.
The out-of-order superscalar pipeline is also modified to disallow memory prediction, reordering of branches, and assume scratchpads are used instead of caches.  
This allows the execution of traces to run predictably for each different exit (branch mispredict) within a trace.
They show an improved throughput for most programs when compared to a simple in-order CPU model, and studied the effects of trace sizes in order to balance the main path execution time against the costs of side exits 

% There are issues with this work, as the assumption is the ability to find the WCEP when doing the trace scheduling (how do you obtain numbers for the basic blocks with Out of Order execution, and what if program itself is so complex that the analysis is nearly infeasible?). 
% But the idea of using static branch prediction in combination with WCET could be leveraged. 
% The delayed scheduling of traces to flush the pipeline could be a huge penalty for programs with small tasks that execute frequently. 
% Reducing this delay is thus a trade off between performance of traces vs amount of state to keep to obtain the performance.

\subsection{VLIW architectures}
\label{sec:RTVLIW}
VLIW machines, like superscalars, issue multiple instructions at a time to exploit ILP.
However, unlike superscalrs, they rely on compiler to utilize ILP and determine the instructions issued.  
This helps in the predictability of the software because the hardware does minimum reordering or stalling.
 
Yan et al.~\cite{Yan2008VLIW} studied the predictability of VLIW machines, and propose changes to the architecture and compiler to improve the predictability. 
They found that although most of the data dependency is scheduled away by the compiler, there are still several factors that limit the predictability on the hardware. 
First, memory access latency will still cause instructions to stall from the architecture. 
Since statically it is not known whether a memory access is a hit or a miss, the hardware still needs to check and stall for it. 
Second, data dependency still exists across compilation units, so the hardware still supports basic data dependency checking to handle those dependencies.
A compilation unit could be a basic block, a loop, a procedure or a region~\cite{Hank:1995:RCI:225160.225189}. 
Finally, if the VLIW uses branch prediction, there is still the need for handling of mis-prediction etc.

As VLIW machines heavily utilize the compiler to improve performance, they propose several compiler techniques to compile programs that lend themselves to better WCET for VLIW architectures. 
First, they use the single-path paradigm proposed by Puschner and Burns~\cite{Puschner:2002:WTP:882515.885528}, and eliminate all non-loop backwards branches with full if-conversions~\cite{Allen:1983:CCD:567067.567085}.  
To mitigate the performance penalty of single-path programming with aggressive hyperblock scheduling~\cite{Mahlke:1992:ECS:144953.144998} to exploit the ILP from VLIW architectures. 
For the data dependencies across compilation units, they use code padding to ensure the execution time is consistent across different paths.    
This will enable easier WCET analysis. 
This work minimally deals with instruction caches, but does not account in the effects of data cache. 

\subsection{Multithreaded Pipelines}
\subsubsection{Thread Scheduling}
With explicit hardware multithreading, the scheduling policy plays a huge role in the predictability of the architecture.
Kreuzinger et al.~\cite{Kreuzinger2000RTmultithread} evaluated the use of different real time scheduling schemes to schedule hardware threads to handle external events. 
They evaluated fixed priority preemptive (FPP), earliest deadline first (EDF), least laxity first (LLF) and guaranteed percentage (GP), which is essentially time sharing the pipeline. 
The architecture used for evaluation is a java multi-threaded superscalar pipeline with four threads~\cite{Kreuzinger2003multithreadeventhandle}. 
A hardware priority manager is implemented to facilitate the scheduling of threads. 
All real-time threads registers its real-time requirements during initialization stage to the priority manager. 
When the external event occurs, the priority manager schedules the corresponding interrupt service thread, and starts assigning priorities based upon the real time requirements. 
The evaluation criteria to compare scheduling policies is the throughput of the processor. 
The conclusion of the report is that in order to maximize multiple threads on a superscalar machine, the scheduler should try and keep as many threads active as long as possible to leverage thread level parallelism and hide more latencies of pipeline stalls. 
Thus GP does the best because it schedules different active threads each cycle until their percentage runs out, thus it keeps threads alive as long as possible. 
The idea of using hardware threads to service interrupts is novel because of the low overhead to switch contexts. 
By giving the interrupt service routine thread priorities, it may be possible the bound the execution time of higher priority threads. 
Although the dynamic thread scheduling can cause execution time bounds to be imprecise from the effects of timing interference across threads.  
  
El-Haj-Mahmoud et al.~\cite{El-Haj-Mahmoud2005VirtualMultiprocessor} proposed a statically scheduled multithreaded architecture called the Real-Time Virtual Multiprocessor(RVMP).   
The idea of a virtual processor is a slice of time on the process. 
The RVMP extends an in-order 4-way superscalar processor to support the partitioning of the pipeline in \emph{space} and \emph{time}.  
In the \emph{space} dimension, the resources of the superscalar can be partitioned to different threads. 
In the \emph{time} dimension, the superscalar resources are time shared, and different threads are scheduled to utilize the resources at different times.  
The hardware extensions to the superscalar pipeline prevent interference between the virtual partitionings.
Scratchpads are employed for predictable memory access latencies, although they assume all accesses go to the scratchpad. 
It is unclear how accesses to shared resources, in particular main memory are dealt with.
A static round-based schedule of the thread execution is constructed to account for the real-time requirements of each thread.
The static schedule utilizes the flexibility of the different time and space partitioning options to allow threads with higher utilization more access to the pipeline. 

\subsubsection{Simultaneous Multithreaded Architectures} 
\label{sec:RTSMT}
Simultaneous Multithreaded Architectures (SMT) attempt to exploit both instruction-level and thread-level parallelism by dynamically scheduling multiple hardware threads onto a multi-way pipeline. 
Each cycle, instructions from different threads can be fetched simultaneously to fully utilize the pipeline.
The dynamic scheduling and aggressive speculation techniques render SMTs almost impossible to use for real-time systems.  
However, several proposals involved slight modifications to architecture to create a \emph{WCET-aware} SMT to be used for real-time systems.  

Barre et al.~\cite{Barre2008RTSMT} proposed to assign one explicit hardware thread with the highest priority. 
That thread, called the real-time thread, gains access to any resource whenever it is scheduled. 
Any other thread that is currently occupying the resource will be preempted, and later replayed when the real-time thread is not using it.
The modifications to the SMT include additions to allow the preemption, and also the partitioning of any resource that needs to be shared. 
This gives the highest priority thread the illusion that it has the whole superscalar pipeline to itself, reducing the execution time analysis of the real-time thread to the equivalent of a superscalar architecture. 
Currently the cache effects and branch prediction are listed as future work.  

Hily et al.~\cite{Hily1999} showed that out-of-order execution may not be as cost effective as in-order execution on SMT machines.
Thus, Uhrig et al.~\cite{SaschaUhrig2005SMT} proposed a similar concept to Barre et al.~\cite{Barre2008RTSMT}, except for an in-order executed superscalar.
Mische et al.\cite{Mische2008SMT} expands this to allow more than one real-time thread running on the SMT architecture.
They do so by essentially time-sharing the highest priority thread slot amongst the real-time threads.  
The time-sharing schedule is statically constructed to ensure that the real-time threads still provide reasonable WCET guarantees. 
This architecture uses instruction scratchpads without data scratchpads, and no branch predictors, as the branch penalty can be filled with executions from other threads.   
Some issues do arise with the contention of memory access, as it is difficult to partition memory accesses between hardware threads.
Contention between the high priority thread slot and other thread slots are resolved by alerting the memory controller from earlier stages in the pipeline that a high priority thread will issue a memory instruction. 
This way the memory controller can hold off service to the lower priority memory accesses and wait for the high priority access to come.     
Contention between the real-time threads on the high priority slot however are not resolved. 
 
% \bigskip
% Although real time SMT proposals have a lot of compelling features, it
% still lacks a formal WCET analysis. All of the proposals have just
% used measured benchmarks to prove performance, and assumed that
% because the high priority thread runs as if no other thread is
% running, so static analysis is possible. In reality, WCET for a
% superscalar processor already contains pessimistic results. Although
% the lack of a branch predictor and in order execution may aid in
% this. Also, in this line of work still only one explicit hardware
% thread slot gets real-time performance, while the other threads'
% performance will have a non-continuous penalty, which is hard to
% categorize. 

\subsection{Others}
\subsubsection{Virtual Simple Architecture}
\paragraph{Virtual Simple Architecture (VISA): Exceeding the Complexity Limit in Safe Real-Time Systems~\cite{Anantaraman2003VISA}}

\begin{itemize}
\item What is the background of this work? What is the motivation?\\
	Modern processors include feature like deep pipelining, dynamic branch prediction, multiple instruction issue, multithreading, out-of-order execution.
	These features increase average-case performance and energy-efficiency.
	However, deriving tight and safe bounds on the WCETs of tasks on such processors is intractable. 
\item What is the main goal?\\
	The goal is to profit from architectural advances, which are intractable for WCET analysis, but improve throughput and energy consumption, without sacrificing safety.
\item What did they do to achieve this goal?\\
	They propose virtual simple architectures (VISA).
	A VISA specifies the timing of a hypothetical simple pipeline that is amenable to safe and tight WCET analysis.
	A microarchitecture should have a simple mode in which it conforms to the VISA specification.
	Outside of this safe mode, it can use arbitrary performance-enhancing features.
	Splitting tasks into subtasks allows to profit from the advantages of the modern processor in the average case:
	Tasks are speculatively executed in high-performance mode, as long as execution time is within WCET bounds determined under the VISA specification.
	Only if this is ``close to become unsafe'' is execution switched to safe mode, which is then still able to meet the deadline.
	In most executions, this will not happen, and one can benefit from the architectural improvements in terms of energy efficiency and performance.
	The paper specifically studies the use of dynamic voltage scaling (DVS):
	Because tasks are typically executed much faster in high-performance mode the operating frequency can be reduced to safe energy.
\item How do they evaluate their approach? Does it achieve the goal? Do they compare it with other work?\\
	They use six benchmarks from the C-lab real-time benchmark suite.
	These benchmarks are then split into five to ten subtasks.
	Simulated execution times in high-performance mode are between three and six times higher than in safe mode.
	Accordingly, the processor frequency can be significantly reduced by reducing frequency to 125 to 225 MHz in high-performance mode as opposed to 375 to 600 MHz in safe mode.
	This yields energy savings between 10\% and 70\%.
\item What other work is listed as future work?\\
	Instead of using slack to save energy, one could execute other soft-real-time and non-real-time tasks.\\
	Another direction would be to investigate methods for ensuring VISA-compliance that do not require a simple mode of operation, but rather comply by design.
\end{itemize}
Additional questions:
\begin{itemize}
  \item What are the limitations/assumptions of this work?\\
  	Because of overheads due to switching between high performance and safe mode and a limited number of checkpoints, there has to be some slack to begin with.
  	In other words, a system that is barely schedulable according to the WCET bounds cannot profit from the described approach.\\
  	A ``real'' safe architecture may be run at a slightly higher clock frequency\footnote{The paper assumes 50\%.}, than the safe mode of a high-performance architecture, rendering some task sets unschedulable on the high-performance architecture, that would have been schedulable on a ``real'' safe architecture.
  \item Which parts of a system/design process are modified by this work? (e.g. hardware (which feature?), WCET analysis, scheduling, compiler, programming language, \ldots)\\
  	Hardware: entire system. Task execution has to be split up into sub-tasks.
\end{itemize}



\subsubsection{Java Architectures}
\label{sec:RTJava}
Schoeberl designed the Java Optimized Processor (JOP)~\cite{jop:wcet}
to propose using Java for real time embedded systems. The design of
JOP includes a two level stack cache architecture
~\cite{jop:stack}. Instead of using a large register file to store the
stack like in PicoJava\cite{McGhan1998PicoJava}, it only uses two
registers to store the top two entries of the stack (Register A and
Regiser B). Leveraging the stack based architecture of JavaVM,
whenever an arithmetic operation occurs, the result is always stored
back to the top of the stack (Register A). Any push or pop operation
simply results in a shift of values between the two registers and the
stack cache, which only requires one read and one write port for the
memory. This architecture does have any data hazards and has very few
pipeline stages (no need for an explicit commit/writeback
stage). Because of the few pipeline stages, it only has a small branch
delay penalty. Also, all bytecode on JOP is translated into a fixed
length microcode. Each microcode executes in a fixed amount of cycles
(mostly 1 cycle) independent of its surrounding instruction, thus the
WCET analysis only requires a lookup table of bytecode translated into
microcode, and it knows the execution time. 

\subsubsection{MCGREP}
Whitham and Audsley's MCGREP~\cite{whitam:06:mcgrep} use programmable microcode to accelerate hotspots that are otherwise too slow.

\subsubsection{PRET}
Craven et. al~\cite{Craven_PRET_implementation_2010} implements PRET thread interleaved pipeline as an open source core using OpenFire 

\section{Memory-Focused Techniques}

Metzlaff et al.~\cite{Metzlaff2008MethodcacheSMIT} (also Theo Ungerer)
also proposed a method to deal with instruction caching. First they
propose a separate scratchpad for each thread. Then they used method
cache~\cite{Kirner2007ModelFunctionCache} with scratchpads and give
priority to the top thread when a filling is needed. If the thread is
stalled when a method is being filled into the scratchpad, then it
seems there would be no data cache access conflict. By using a SMT
machine, if the high priority thread is filling its cache, the lower
priority threads could still fill the pipeline and gain utilization
and throughput.

\subsection{Caches}
The basic block of caches is a cache line. A cache line can contain
$B$ words. When the cache is filled, the whole line is filled. When a
replacement is needed, the whole line is replaced. The cache can be
$N$-way associative, where N is the number of lines in each set. If a
cache is 1-way associative(direct mapped) with size $S$, then it has
$M = S \div B$ sets, where $M$ is the number of sets in the cache. A
cache size therefore is determined by $S = M \times N \times B$.

If a cache is direct mapped, it does not need replacement policy,
since you always know which line needs to be evicted. However, if
the cache is more than one way, then a replacement policy is needed to
determine which way is to be replaced.

\textcolor{red}{Talk about tradeoff in cache size/cache way etc}
\subsubsection{Unified vs Separate Cache} Typically a cache of an architecture is
structured in 2 ways. A \textit{Unified Cache} is one where
instruction and data are in the same cache. A \textit{Separated Cache}
is one where instruction and data are on separate caches, each with
independent state information. A unified cache introduces interference
between instruction and data accesses. Thus, this complicates the WCET
analysis even more because data caches are harder to model or
determine the value. Thus, Separate caches for instruction and data
should be used to help obtain a more precise and simpler analysis.

\subsubsection{Replacement Policies} 
To read: \cite{Heckmann2003processor}

\paragraph{Timing Predictability of Cache Replacement Policies~\cite{Reineke07TimingPredictability}}
\begin{itemize}
\item What is the background of this work? What is the motivation?\\
	It had been observed in previous work, e.g. \cite{Heckmann2003processor} that cache analyses for LRU (least-recently-used) were more precise than analyses for other policies like FIFO and PLRU. 
	However, it was not clear, whether this was because analyses of FIFO and PLRU were less developed than those of LRU, or whether LRU was simply more ``predictable'' than other policies, and analyses of similar precision were unattainable for FIFO and PLRU.
\item What is the main goal?\\
	To study the predictability of cache replacement policies and to explain empirical observations.
\item What did they do to achieve this goal?\\
	Cache analyses have to cope with uncertainty. Sources of uncertainty are, among others:
	\begin{itemize}
      \item The cache contents when a task is started are usually unknown.
      \item If the address of a data accesses cannot be precisely determined, this introduces uncertainty about the cache state.
      \item Preempting tasks may change the cache state at preemption points.
    \end{itemize} 
    These sources of uncertainty are independent of the particular cache organization.
    The precision of cache analyses is therefore determined by the speed at which information about the cache state is obtained.
	They defined two predictability metrics that determine how quickly knowledge about cache hits and misses can be (re-)obtained for a particular cache replacement policy. These metrics are independent of any particular cache analysis, i.e. they mark the precision limit for any possible cache analysis.
	They go on to evaluate the policies LRU, FIFO, PLRU, and MRU under these metrics.
	The values of the different policies confirm the expectation that LRU is significantly more predictable than other policies.
	It, however, also revealed potential for improvement in existing analyses of PLRU and in particular FIFO.
\item How do they evaluate their approach? Does it achieve the goal? Do they compare it with other work?\\
	They compare their results to previous empirical evidence in cache analysis, which was based on specific static cache analyses.
\item What other work is listed as future work?\\
	To investigate the precision of cache analyses and to develop new cache analyses that are optimal w.r.t. the metrics (partly solved in subsequent work).
\end{itemize}
Additional questions:
\begin{itemize}
  \item What are the limitations/assumptions of this work?\\
  	A limitation of the metrics is that they assume complete uncertainty, whereas partial information about the cache state is often available.
  \item Which parts of a system/design process are modified by this work? (e.g. hardware (which feature?), WCET analysis, scheduling, compiler, programming language, \ldots)\\
  	This work does not propose a new feature. However, it recommends the use of LRU replacement in caches, which affects the cache itself and the WCET analysis.
\end{itemize}



\subsubsection{Instruction Cache}

\paragraph{A Time Predictable Instruction Cache for a Java Processor~\cite{jop:jtres_cache}}
\label{sec:func_cache}
Schoeberl~\cite{jop:jtres_cache} was the first to propose Method
Caches (Called function cache in\cite{Metzlaff2008MethodcacheSMIT})
for the instruction cache. The idea of a method cache is that the
granularity of the cache block replacement size is now not a cache
block, but a method instead. When a method is called, the cache
content is loaded with that method. The cache could contain a single
block, two block, or variable blocks to load the caches. The different
block size gives different trade offs. The single block gives perfect
predictability, but bad performance, since every time a method is
called or returned to, it needs to be loaded. Not only that, some
methods are loaded only to execute a few instructions before calling
another method, resulting in high overhead. The two block method cache
can cache two methods. The replacement policy would be LRU. This gives
better performance, because on a return to the caller method, the
cache will yield a hit (if no other methods were called). The
performance of the two block is about two times better than a single
block. To further improve, the cache could be split into several
blocks. A method could have the size of several blocks, but some
methods may only need one block. When a method is loaded, it occupies
the amount of blocks it needs. When a method is evicted, all blocks it
occupies are evicted. A pointer stores the next block to replace or
load methods in. This could give better performance because depending
on the block size you chose and the method sizes, you could more
efficiently using the cache blocks. But if there are a lot of blocks,
it could also increase the WCET difficulty.

\begin{itemize}
\item What is the background of this work? What is the motivation?\\
	Standard cache organizations improve the average-case execution time but are difficult to predict for WCET analysis.
	\JR{Unfortunately, this is really the only motivational part I could find in the paper.}
\item What is the main goal?\\
	A cache organization that is amenable to simpler and more accurate WCET analysis.
	Performance should not be completely neglected.
\item What did they do to achieve this goal?\\
	Schoeberl proposes the method cache.
	Instead of caching memory blocks of fixed size as in a traditional cache organization, the method cache caches entire methods.
	This simplifies cache analysis as there are fewer points in a program at which the cache is filled: this may only happen on calls and returns.
	It is claimed that tag memory and address translation are not necessary in a method cache. \JR{Something like a tag memory is still necessary to remember which methods are cached.}
	Several variants of the method cache are introduced: single method cache, two block cache, variable block cache.
	The single method cache always caches the current method and nothing else.
	The two block (why not method?) cache can cache up two methods. It can easily be implemented with LRU replacement.
	To reduce the waste of cache memory the variable block cache is introduced, which can store more than two methods.
	The variable block cache can store a fixed number of blocks, which are similar to cache lines in traditional designs.
	In this organization, LRU replacement cannot be realized.
	Two replacement policies are proposed: next block (FIFO) and stack oriented.
	WCET analysis is briefly sketched.  
\item How do they evaluate their approach? Does it achieve the goal? Do they compare it with other work?\\
	The new cache design is evaluated by simulations of a single application from one node of a distributed motor control system.
	For one cache size (2KB) and the variable block cache with 32 blocks, the method cache outperforms a direct-mapped cache of equal size.
	The improvement in predictability is not quantitatively evaluated.
	It is argued, that the single method cache is more predictable than the two-block cache which is in turn more predictable than the variable block cache.
	WCET analysis of the variable block is ``nevertheless simpler than that with the direct-mapped cache''.
	Filling the cache only on method invocations and returns removes potential competition with data accesses.
\item What other work is listed as future work?
\end{itemize}
Additional questions:
\begin{itemize}
  \item What are the limitations/assumptions of this work?\\
  	It deals with instruction caches only.
  \item Which parts of a system/design process are modified by this work? (e.g. hardware (which feature?), WCET analysis, scheduling, compiler, programming language, \ldots)\\
  	The java processor and the WCET analysis.
\end{itemize}

\subparagraph{Static Analysis of the Method Cache~\cite{Kirner2007ModelFunctionCache}}
\begin{itemize}
\item What is the background of this work? What is the motivation?\\
	The method cache was introduced as a time-predictable instruction cache.
	However, only simple variants had been statically analyzed before.
\item What is the main goal?\\
	To develop a precise static analysis of the method cache.
\item What did they do to achieve this goal?\\
	They adapted the techniques of Ferdinand et al. to method caches and introduced the notion of local persistence analysis.
	Local persistence analyses determine whether a memory reference within a scope (therefore local) can cause more one cache miss.
\item How do they evaluate their approach? Does it achieve the goal? Do they compare it with other work?\\
	The new analyses is evaluated on a small synthetic benchmark consisting of ``a few functions, two loops and a [sic] if statement''.
	They compare the WCET bound obtained with the proposed analysis to the WCET bounds obtained with two previous analyses and a measured execution time.
\item What other work is listed as future work?\\
	To finish the implementation of the analysis and to compare it to existing analyses of set-associative caches on a collection of benchmarks.
\end{itemize}
Additional questions:
\begin{itemize}
  \item What are the limitations/assumptions of this work?\\
  	According to Martin Schoeberl, this analysis is incorrect.
  	The abstract updates are not described in the paper.
  	With the given domains, sound updates should not yield any ``non-trivial'' classifications.
  \item Which parts of a system/design process are modified by this work? (e.g. hardware (which feature?), WCET analysis, scheduling, compiler, programming language, \ldots)\\
  	WCET analysis.
\end{itemize}


\paragraph{Trace Cache}

\subsubsection{Static Cache Locking}

\paragraph{Low-Complexity Algorithms for Static Cache Locking in Multitasking Hard Real-Time Systems~\cite{Puaut02}}

\begin{itemize}
\item What is the background of this work? What is the motivation?\\
	Schedulability analysis depends on safe bounds on the WCETs of tasks.
	Caches have a strong influence on the execution times of tasks.
	It is hard to statically predict execution times in the presence of caches:
	Such an analysis has to consider both intra- and inter-task interferences. 
	Intra-task interferences occur when different memory blocks of the same task compete for cache blocks.
	Inter-task interferences, imply a so-called cache-related preemption delay, where a preempting task's memory blocks cause cache reloads in the preempted task.
	Analyses for both intra- and inter-task interferences have been developed.
	An alternative to analyzing these interferences is eliminating them.
	Task/cache partitioning and cache locking have been proposed for this purpose.
	In cache partitioning a part of the cache is reserved for a particular task.
	This eliminates inter-task interferences, but intra-task interferences remain.
	Similarly, the data layout of tasks can be adapted in such a way that different tasks do not interfere in the cache.
	This is known as task partitioning.
	Finally, cache locking has been proposed to eliminate both inter- and intra-task interferences.
	In cache locking, the contents of the cache are loaded and locked at fixed times.
	In \emph{static cache locking} they are fixed at system start for the whole system lifetime, whereas the contents may be changed in \emph{dynamic cache locking}, for instance at preemptions.
	As what is being locked into the cache is decided statically, it is usually~\footnote{Cache locking techniques could, in principle, mimic the dynamic behavior of a cache.} easy to predict the memory access times.
	
\item What is the main goal?\\
	To explore the use of static cache locking of instruction caches in multitasking real-time systems.
	
\item What did they do to achieve this goal?\\
	They propose two algorithms for static cache locking.
	The first algorithm minimizes the utilization of a task set.
	Typically, lower utilization results in better schedulability.
	Sufficient (and in some cases necessary) schedulability conditions based on the utilization exist for different scheduling regimes.
	For fixed-priority scheduling, necessary and sufficient response-time analysis (RTA) methods have been developed.
	The second proposed algorithm tries to increase schedulability by minimizing interference between different tasks, which is considered in RTA.
\item How do they evaluate their approach? Does it achieve the goal? Do they compare it with other work?\\
	They compare their two static cache locking approaches with a regular unlocked cache.
	They analyze the WCET in case of an unlocked cache using a (at the time) state of the art instruction cache analysis~\cite{Mueller00}.
	To estimate the cache-related preemption delay (CRPD) they assume that all blocks of a task have to be reloaded after a context switch (At the time a more accurate analysis of CRPD was available: Lee et al.~\cite{Lee96}. More precise analyses based on the original approach have been developed recently).
	They analyze two task sets, each consisting of four tasks. For these tasks, they choose a period that would result in a utilization of 1.3 if no cache would be employed. 
	For a number of cache configurations of varying associativity and size, they report the utilization determined by static analysis.
	In addition to reporting utilization, they also indicate whether the task set was deemed schedulable by schedulability analysis.
	For small associativities, the unlocked cache significantly outperforms statically locked caches in terms of utilization and schedulability.
	For higher associativities, the static cache locking techniques perform better, as they are more flexible in their decisions of which instructions to lock.
	On the other hand, the static analysis of the unlocked cache is less precise for higher associativities\footnote{This is surprising and not explained.}.
	For associativities greater than 4 or 8, static cache locking outperforms static analysis of unlocked caches.
	
\item What other work is listed as future work?\\
	To investigate average-case performance of static cache locking.
	The proposed algorithms assume a fixed ``worst-case execution path''. In future work, they would like to investigate how sensitive the algorithms are to this path.
	They would like to compare their algorithms to genetic algorithms proposed by Campoy et al.
	For large programs, they plan to explore dynamic cache locking strategies.
\end{itemize}
Additional questions:
\begin{itemize}
  \item What are the limitations/assumptions of this work?\\
  	The cache supports locking contents.
  \item Which parts of a system/design process are modified by this work? (e.g. hardware (which feature?), WCET analysis, scheduling, compiler, programming language, \ldots)\\
  	Compiler, WCET analysis.
\end{itemize}



\subsection{Scratchpads}
Scratchpads (ARM refers to this as Tightly Coupled Memory - TCM) are a
form of fast access memory, except the allocation of the data on the
scratchpad is controlled by software, where as caches are controlled
in hardware. This is done often by mapping a memory space as the
scratchpad space, where all accesses to that space will go to
scratchpad and all other accesses will go to memory. This allows more
precise WCET analysis~\cite{Wehmeyer2005SPM}~(in the absence of caches)
because based upon what memory location is being accessed, you can
categorize the access latency (of course one difficulty in WCET is
knowing what the actual memory location is). There are two allocation
schemes that are employed by the compiler. A \textit{static allocation
  scheme} allocates the data before the program runs, and the
allocation doesn't change throughout the program (see
\nameref{sec:SPM_static_allocation}). A \textit{dynamic allocation
  scheme} can change the mapping of data on the scratchpad during run
time in attempt to increase performance (See
\nameref{sec:SPM_dyn_allocation}).


\subsubsection{Static allocation schemes}
\label{sec:SPM_static_allocation}
Static allocation schemes allocate the content on the scratchpad a
priori, and the content stays the same throughout the whole
execution. Most static scratchpad allocation schemes
(\textcolor{red}{Find some citations}) use some sort of heuristic to
find the most commonly executed instructions or data structures, and
then allocate them statically on the scratchpad to improve the ACET
(average case execution time). Suhendra et
al.~\cite{Suhendra2005WCETSPM} (Abhik Roychoudhury) was the first to
propose using static data scratchpad allocation to improve the worst
case execution time. They first identified the difficulty in
constructing such an algorithm that optimally allocates data to
improve the WCET because once data is allocated on the SPM, the worst
case execution path will change. Also, it's possible that the WCEP
that's resulted from static analysis is actually infeasible, so the
detection of infeasible task is also needed. As a result, they
proposed a greedy method that allocates contents of the SPM in attempt
to reduce the WCET. They first assume that no allocation is done for
any memory location. They find the WCEP, and allocate the most used
data block (not too sure of what size, does it depend on the data
structure? Is it fixed size?)  from that path onto the SPM, then
reiterates the algorithm to find the new WCEP. This iterates until the
SPM space is exhausted. Note that is is sub-optimal because allocated
content will not be reconsidered. This means that if the first WCEP
and the second WCEP don't share that data structure, the first
allocation does not contribute to improving the WCEP.

Patel et al.~\cite{Patel2008PRETSPM} proposed another static
allocation scheme based on \cite{Suhendra2005WCETSPM}, except that the
allocation criteria wasn't to minimize the WCET, but to meet all
deadlines in the program. The greedy approach worked by synthesizing
timing constructs (deadline blocks) into test programs. The algorithm
works by identifying the deadline blocks that miss their
deadline. Then a profile is constructed based on the number of
accesses to a data block in the missed deadline blocks. Based on the
profile, a data block is selected to be allocate on the scratchpad,
and then the algorithm reiterates itself, until either all deadlines
can be guaranteed to be met, or if the SPM exhausts its space. In the
first case, the remaining space can be optimized by another method. In
the second case, the program is deemed un-schedulable, and the
deadlines cannot be met. 

\subsubsection{Dynamic allocation schemes}
\label{sec:SPM_dyn_allocation}

\subsection{Caches vs. Scratchpad}
Puaut et al.~\cite{Puaut2007SPMvsCache} did a comparison of locked
caches and scratchpads. Some architectures support locking cache lines
and software controlled loading of content into the cache. Puaut
showed that the difference between using locked caches and scratchpads
was minor. Most benchmarks provided similar WCET estimates. The
differences stem from the granularity of blocks. For locked caches,
the basic allocation unit is a cache line. Thus, it's independent of
the basic block size of the allocation. However, this results in
the \textit{pollution} of locked content. \textit{Pollution} occurs
when a cache line that's locked contain words that aren't part of the
allocation scheme, but simply locked in because it belonged to the
same cache line. Also, depending on the associativity of the cache, a
cache line that should be locked could possibly be in conflict with
another cache line that's also locked, and thus lose its ability to be
locked in the cache. For scratchpads, the basic allocation is only
determined by the allocation scheme. However, if the basic allocation
block is big, it's possible that due to the fragmentation, a big
allocation block doesn't fit in the scratchpad at the end. In the
paper, this resulted in a 30\% drop in the on-chip access ratio of
total memory accesses. 

\paragraph{Scratchpad Memories vs Locked Caches in Hard Real-Time Systems: A Quantitative Comparison~\cite{Puaut2007SPMvsCache}}

\begin{itemize}
\item What is the background of this work? What is the motivation?\\
	Caches are used to bridge the increasing performance gap between the processor and the off-chip memory.
	Allocation and deallocation of memory blocks from the cache is managed by hardware in a transparent manner to the programmer and the compiler.
	In hard real-time systems, caches are a source of unpredictability.
	A lot of progress has been made to statically predict cache behavior.
	Due to lack of documentation about the employed replacement policies, this work is not always applicable.
	In addition, such analyses are relatively pessimistic for some replacement policies, such as pseudo round-robin, pseudo-LRU, or random replacement.\\
 	Many processors allow to lock (also: freeze) the contents of the cache. Static and dynamic cache locking techniques have been developed.
 	An alternative to caches is scratchpad memory (SPM).
 	In contrast to caches, scratchpads are under software-control.
 	The compiler or programmer needs to allocated code and/or data to the scratchpad memory.
 	Significant work has been done to develop allocation techniques for SPMs, most aimed at reducing average-case execution time.
 	Only one previous paper~\cite{Suhendra2005WCETSPM} was concerned with WCET-oriented \emph{static} scratchpad allocation.
\item What is the main goal?\\
	To develop a dynamic WCET-oriented instruction scratchpad allocation and cache-locking algorithm.
	This algorithm should minimize the WCET estimate.
	The second goal is to quantitatively compare cache locking with scratchpad allocation.
\item What did they do to achieve this goal?\\
	Their scratchpad allocation/cache locking algorithm proceeds in two steps:
	\begin{enumerate}
      \item Selection of reload points
      \item Selection of on-chip memory contents
    \end{enumerate}
    In the first step, a subset of the loop pre-headers is selected as potential reload points.
    Only at these points, will the algorithm consider to load memory contents into the scratchpad/cache.
    The second step iteratively and greedily decides which blocks to allocate to the fast memory (scratchpad/cache).
    To decide which blocks are most beneficial to improve the WCET estimate, WCET analysis is performed.
    The WCET analysis determines execution frequencies of basic blocks that maximize the execution time of the task.
    These execution frequencies are taken into account when deciding which block to allocate to the fast memory.
    As allocations to the fast memory may change the execution frequencies maximizing execution time, WCET analysis has to be performed again after allocating blocks.
    The algorithm allows to trade off the number of costly WCET analysis invocations and the quality of the allocation.
\item How do they evaluate their approach? Does it achieve the goal? Do they compare it with other work?\\
	The paper focuses on comparing scratchpad allocation with cache locking.
	It does not compare the (worst-case) performance of unlocked caches with that of locked caches or scratchpads.
	The two possibilities are compared regarding the WCET estimates obtained with the Heptane WCET analysis tool.
	Five benchmarks, four from the M�lardalen benchmark suite and one from the UTDSP benchmark suite are used.
	In their initial analysis, the differences between WCET estimates for locked caches and scratchpads are not very large:
	sometimes locked caches outperform scratchpads, sometimes vice-versa.
	This is linked to the cache block size and the basic block size.
	Large cache blocks lead to pollution, where unimportant instructions are locked, as cache blocks can only be locked as a whole.
	Depending on the alignment of basic blocks with cache blocks, this problem is more or less severe.
	As the locking of SPM contents is performed at the level of basic blocks, large basic blocks can be problematic:
	they cause fragmentation, as very large basic blocks might not fit onto the scratchpads.
	Fragmentation could be a more significant problem with large data structures such as large arrays.
\item What other work is listed as future work?
\end{itemize}
Additional questions:
\begin{itemize}
  \item What are the limitations/assumptions of this work?\\
  	Splitting basic blocks to reduce fragmentation.
  	Allocation of data.
  \item Which parts of a system/design process are modified by this work? (e.g. hardware (which feature?), WCET analysis, scheduling, compiler, programming language, \ldots)\\
  	Compiler, WCET analysis.
\end{itemize}


\bigskip
\textbf{Preemption} - Although scratchpads are more predictable, the
hardware replacement policy of caches could be advantageous,
especially for systems that have preemption of tasks. Locked caches
can lock content in the cache, and when a preemption occurs, the
hardware can still take advantage of a hardware replacement for lines
that aren't locked, while still maintaining a certain amount of
predictability when the original task resumes (because of the locked
content that aren't replaced). For statically allocated scratchpads,
the allocation scheme needs to take into account the preempted task,
and allocate space for that task. This will reduce the available
allocation space for all other tasks. For dynamically scheduled
schemes, it will be extremely difficult to find reload points to
adjust the scratchpad. If a preempted task loads its optimal content
when it preempts a task, it's unclear what to load back for the
original task, and when to load it. There needs to be some sort of OS
that keeps track of what was loaded off the scratchpad. It is unclear
what overhead this could result in. (\textcolor{red}{Read papers about
  this} \JR{Write paper about how to do this ;-)})

\subsection{DRAM}

\paragraph{Predator: A predictable SDRAM memory controller~\cite{Akesson2007CODES,Akesson2009DSD,Akesson2010}}
\begin{itemize}
\item What is the background of this work? What is the motivation?\\
   Contemporary multi-processor SoCs (systems-on-chip) feature a large number of intellectual property components, which communicate through 	shared memory. Some of these IPs have hard real-time requirements. The memory traffic generated by the different components is dynamic and not fully known at design time. Standard DDR2 SDRAM memory controllers schedule the requests of the different components dynamically. Predicting the execution time of a particular component in such a system is difficult, because
   \begin{itemize}
     \item the time to service a request depends on past requests,
     \item in particular, it depends on past requests by other requestors
   \end{itemize}
   Due to interference on shared resources, verification complexity of real-time requirements increases dramatically on multi-processor systems.
   If the behavior of different applications can be completely isolated, they can be verified in isolation.
\item What is the main goal?\\
	The first main goal is to provide a memory controller design that provides a guaranteed minimum bandwidth and a maximum latency to each of the requestors (\emph{predictability}) independently of the behavior of other requestors. It should efficiently utilize the memory chip. The second main goal is to provide complete isolation of requestors, i.e. the behavior of one requestor should not influence the service experienced by other requestors (\emph{composability}, this part occurs only in~\cite{Akesson2009DSD,Akesson2010}).
\item What did they do to achieve this goal?\\
	The predictability goals is accomplished in two steps:
	\begin{itemize}
    	\item First, a set of of read and write groups, with corresponding static sequences of SDRAM commands, are determined. These groups determine the minimum request size, which can be, for instance, the size of a cache line. Longer groups will increase memory efficiency, as long as no unnecessary data is fetched. On the other hand, longer groups will increase latency.
    	\item Secondly, a Latency-Rate scheduler (in this case a Credit-Controlled Static-Priority (CSSP) scheduler) is employed to provide the different requestors with the requested service independently of the behavior of the other requestors. CSSP allows to decouple allocated latency from rate, in contrast to TDMA approaches.
    \end{itemize}
    The composability goal is achieved by a special front-end~\cite{Akesson2009DSD,Akesson2010}.
    Essentially, the front-end delays each response by the predictable back-end (memory controller) up to its worst-case bound.
    It may also reject new requests, if the request queues would be full, assuming worst-case behavior of the memory controller.
    In this way, interactions between different requestors are completely eliminated.
\item How do they evaluate their approach? Does it achieve the goal? Do they compare it with other work?\\
	The predictable memory controller Predator is evaluated by simulation of a SystemC model. 
	Four hard real-time requestors are mimicked by traffic generators.
	These traffic generators have a combined bandwidth requirement of 660 MB/s. 
	In the simulations, the maximum observed delays for the requestors are recorded.
	These are then compared with the analytical bounds.
	Reassuringly, the analytical bounds are never exceeded during simulation and all requestors receive their allocated rate.
	However, for lower priority requestors, the latency bounds are fairly pessimistic as the worst-case is extremely unlikely.
	In a second experiment, one of the requestors asks for more resources than allocated. The well-behaved requestors are unaffected by this behavior and continue to receive their allocated rate within the analytical latencies.
\item What other work is listed as future work?\\
	Developing an algorithm for automatic generation of memory access groups, given a set of memory timings and a burst size (proposed in \cite{Akesson2007CODES}, done in \cite{Akesson2010}).
\end{itemize}
Additional questions:
\begin{itemize}
  \item What are the limitations/assumptions of this work?\\
  	It seems that the guaranteed latency is rather high. Guaranteed bandwidth is close to the maximum that can be provided. However, having to account for potential refreshes and performing rather long bursts (through all banks), results in long worst-case latencies, in particular for lower priority tasks, as noted in \cite{Paolieri2009ESL}. This can be partly alleviated by a different arbiter, like round-robin or TDMA, which in turn has other disadvantages.
  \item Which parts of a system/design process are modified by this work? (e.g. hardware (which feature?), WCET analysis, scheduling, compiler, programming language, \ldots)\\
  	WCET analysis is simplified.
\end{itemize}


\paragraph{An Analyzable Memory Controller for Hard Real-Time CMPs~\cite{Paolieri2009ESL}}
\begin{itemize}
\item What is the background of this work? What is the motivation?\\
	Multicore processors provide the performance required by current and future hard real-time systems.
	However, due to interferences on shared resources, it is difficult to compute tight bounds on the WCETs of tasks running on such processors.
\item What is the main goal?
	An analyzable JEDEC-compliant DDRx SDRAM memory controller for hard real-time multicores.
	The WCET estimation of a task should be independent of the memory behavior of other corunning tasks (composability).
	Furthermore, the controller and corresponding analysis should be adaptable to arbitrary JEDEC-compliant DDRx SDRAM devices.
\item What did they do to achieve this goal?\\
	They schedule memory accesses hard real-time tasks (HRTs) in a round-robin fashion.
	These accesses are prioritized over those of non hard real-time tasks (NHRTs).
	Memory requests are serviced in bursts, which are interleaved over all banks. \JR{Note, that this approach interleaves accesses belonging to the same request as opposed to our PRET approach.}
	They analyze the maximal delay a memory access can suffer due to a previous access using generic DDR timing constraints.
	Based on this delay they compute the maximum access delay taking into account the round-robin scheduling. 
	The resulting ``upper bound delay'' (UBD) can be used in WCET analysis.
	They do not change the auto-refresh mechanism.
	They propose synchronizing the start of a HRT with the the occurrence of a refresh operation at analysis and runtime.
	\JR{How much this helps w.r.t. to WCET analysis is questionable.}
\item How do they evaluate their approach? Does it achieve the goal? Do they compare it with other work?\\
	They use a hard real-time application from Honeywell, a collision avoidance algorithm to evaluate their approach quantitatively.
	They consider four scenarios: WCET mode $i$, with $i \in \{1,2,3,4\}$. In WCET mode $i$, there are $i-1$ corunning HRTs and one NHRT. For each of these modes they perform simulations (MOET) and a WCET analysis using the UBDs (AMC). They also perform a WCET analysis assuming a private SDRAM for each task (PRMC). In both cases, the WCET analyses account for interferences on on-chip resources. They also vary the size of the caches from 128KB to 8KB. The smaller, the cache, the more memory requests will be generated by the HRTs. The NHRT is configured to constantly access the memory and to thereby interfere in the strongest possible way. In WCET mode 1, i.e. the HRT competes with no other HRTs but one NHRT, the three analyses (MOET, AMC, PRMC) yield very similar results. In WCET mode 4 and 8KB cache, measurements (MOET) show a slowdown compared with PRMC of around 1.3. AMC computes a bound which is about 1.7 times higher than PRMC, indicating an overestimation around 1.3.
	
	They compare their approach with Predator~\cite{Akesson2007CODES}. The first argument in favor of their approach is its applicability to arbitrary JEDEC-compliant DRAM devices. However, this requirement was also lifted in Akesson's PhD thesis.
	They also compare the UBDs provided by Predator with those provided by their own approach. 
	In Predator each HRT is assigned a priority, with decreasing priority, the UBDs rise.
	While the UBD of the highest priority task is slightly better than the UBD of AMC (their approach), the guarantees for lower priority tasks are much worse in Predator.
	This is caused purely by Predator's backend, which could be easily replaced by any Latency-Rate server, such as the one presented in this work.
\item What other work is listed as future work?
\end{itemize}
Additional questions:
\begin{itemize}
  \item What are the limitations/assumptions of this work?
  \item Which parts of a system/design process are modified by this work? (e.g. hardware (which feature?), WCET analysis, scheduling, compiler, programming language, \ldots)
\end{itemize}


\paragraph{Making DRAM Refresh Predictable~\cite{Bhat2010PredictableDRAM}}
\begin{itemize}
\item What is the background of this work? What is the motivation?\\
	DRAM cells leak charge and have to be refreshed periodically to retain their state.
	This is usually done through auto-refreshes issued by the DRAM controller.
	From a task's perspective such auto-refreshes occur asynchronously.
	Refreshes affect timing of memory accesses in two ways:
	\begin{itemize}
      \item by stalling DRAM accesses during refreshes,
      \item by closing DRAM rows, which leads to additional precharges to reopen them.
    \end{itemize}
	Therefore they are difficult to account for in WCET analysis.
\item What is the main goal?\\
	Eliminate interferences between refreshes and memory accesses of tasks, such that WCET analysis can be performed without considering refreshes.
\item What did they do to achieve this goal?
	They execute refreshes in bursts.
	These refresh bursts are scheduled in periodic tasks.
	This way they can be taken into account during schedulability analysis.
	They present two implementations of bursty refreshes.
	One purely software-based, the other relying on the auto-refresh capabilities of the DRAM controller.
\item How do they evaluate their approach? Does it achieve the goal? Do they compare it with other work?\\
	They evaluate their approach by measurements on a DSP and an ARM platform.
	To analyze the effect of refreshes on execution times they deactivate the instruction and data caches.
	This way every memory access goes to the DRAM.
	With standard DRAM auto-refresh they observe a small jitter of less than 0.1\% in the execution times of the bubble sort algorithm.
	Their approach eliminates this jitter completely and improves execution times slightly, by 0.18\% to 2.8\%.
	The overhead incurred by the two bursty refresh mechanisms is between 3 and 16\%.
	They also observe, using measurements of the energy consumption of the DRAM module, savings of about 5\%.
	However, it should be noted, that the overhead incurred by the refresh tasks is not taken into account in these savings.
	They will likely be greater than the savings.
	The savings will also be reduced by the use of scratchpads or caches. 
\item What other work is listed as future work?\\
	They would like to pursue FPGA-based motifications to the DRAM controller to add native support for burst refreshes in hardware.
	Burst refreshes could be overlaid with non-memory based activities.
\end{itemize}
Additional questions:
\begin{itemize}
  \item What are the limitations/assumptions of this work?\\
  	This work considers unpredictability of DRAM accesses due to refreshes.
  	It does not consider variations in access times due to the access history or due to competing requestors.
  	These variations are somewhat different in nature as they can in principle be analyzed in the WCET analysis of a task.
  \item Which parts of a system/design process are modified by this work? (e.g. hardware (which feature?), WCET analysis, scheduling, compiler, programming language, \ldots)\\
  	Scheduling, possibly the DRAM controller
\end{itemize}

Several research groups have introduced ways to mitigate the temporal interference in SMTs by allocating a real-time thread with the highest priority ~\cite{Barre2008RTSMT}, time-sharing the real-time thread ~\cite{Mische2008SMT}, and partitioning the instruction cache ~\cite{Metzlaff2008MethodcacheSMIT}.

\paragraph{Implementing Time-Predictable Load and Store Operations~\cite{whitham:09:timepredLoadStore} }

\begin{itemize}
\item Basic definitions: \\
\textbf{Pointer aliasing:} The same memory location may be referenced using different names (pointers). 
\textbf{Pointer invalidation:} An object in a memory location is moved out from that memory location.  As a result, an alias that points to the object before the move, ends up pointing to an incorrect object. 
\textbf{Whole-program Pointer analysis:} Determine which pointers can point to which variables and storage locations in the entire program.
\item What is the background of this work? What is the motivation?\\
Scratchpad (SPM) memories provide time-predictable accesses for data.  However, the time-predictable SPM allocation strategies for data only support statically allocated data or stack data.  Dynamic data, on the other hand, is only supported by non time-predictable allocation schemes if whole-program pointer analysis identifies every memory operation that could access each variable. 
\item What is the main goal? \\
The objective of this paper is to implement a scratchpad memory management unit that transfers data between external memory and scratchpad memories such that pointer aliasing and pointer invalidation are eliminated. 
This approach lifts some of the restrictions (i.e. eliminate pointers entirely from program code or no support for dynamic data) forced by the limitations of WCET analysis. 
\item What did they do to achieve this goal?\\
They implemented a scratchpad memory management unit (SPMMU) in hardware that performs a mapping from logical addresses (used by the program) to physical addresses (identifying where an object resides).  In addition, the SPMMU performed DMA transfers between the external memory and SPM.  The program dictates when the transfers take place via explicit OPEN and CLOSE commands in the code. The user specifies the base address for the object, the size of the object and the physical address at which the object is being loaded to.  The SMMU then performs the transfer but updates an internal table mapping the logical address to the new physical location of the object.  
\item How do they evaluate their approach? Does it achieve the goal? Do they compare it with other work?\\
They use three approaches to evaluate their work.  
Their first approach simply looks at the hardware cost incurred.  In particular, they observe that since the mapping must take place in the same cycle that the request is made, it must be implemented as combinational logic; hence, a part of the critical path.  They determine this critical path and its effect on the clock frequency. 
The second approach compares their approach for a particular function of a JPEG decode algorithm with a data cache with estimated access latencies.  They show that the SMMU is not as effective as a data cache in ideal conditions, they are much better in the worst case. 
The third approach is a case-study implementation of the entire JPEG decode algorithm with the SMMU. 
\item What other work is listed as future work?
Integrating one of the SPM allocation techniques with the SMMU to determine the where to place OPEN and CLOSE commands. 
\end{itemize}
Additional questions:
\begin{itemize}
  \item What are the limitations/assumptions of this work?\\
The user must specify the size of the object and the physical address. 
They claim that there are algorithms that determine the best physical address, but I haven't read that work yet. 
  \item Which parts of a system/design process are modified by this work? (e.g. hardware (which feature?), WCET analysis, scheduling, compiler, programming language, \ldots)\\
They require the following modifications: 1) hardware, 2) WCET analysis, and 3) compilers to generate the OPEN/CLOSE commands. 
\end{itemize}

% \section{Interconnect}
% 
% \paragraph{Real-Time Control of I/O COTS Peripherals for Embedded Systems~\cite{Pellizzoni08,Bak09}}
% \begin{itemize}
% \item What is the background of this work? What is the motivation?\\
% 	Due to mass production, commercial-off-the-shelf (COTS) peripherals are not only cheaper than custom made systems, but also provide performance orders of magnitude higher (e.g. PCI Express vs. the real-time SAFEbus). It is thus tempting to employ COTS components in real-time systems. However, their unpredictable timing makes their use in such systems difficult: they are typically designed paying little attention to worst-case timing behavior. This paper considers the I/O subsystem. Modern real-time systems may include several high-bandwidth I/O devices. Connecting these devices to the processor through COTS interconnects yields unpredictable delays and may cause deadline misses. In contrast to CPus, where real-time scheduling is common, it is not supported by COTS interconnect systems such as the PCI bus.
% 	
% \item What is the main goal?\\
% 	The goal is to enable the use of COTS interconnect in real-time systems.
% 
% \item What did they do to achieve this goal?\\
% 	They designed a real-time I/O management system. This system consists of a real-time bridge for each COTS peripheral connecting the peripheral to the COTS interconnect, and a reservation controller. The reservation controller decides at any point in time which real-time bridge may access the COTS intereconnect. It can implement a variety of real-time scheduling policies, such as EDF or RM.
% They extend the Real-Time Calculus to determine I/O delay bounds and each bridge's necessary buffer size to guarantee lossless traffic delivery.
% 
% \item How do they evaluate their approach? Does it achieve the goal? Do they compare it with other work?\\
% 	They evaluate their approach through measurements of synthetic benchmarks on a COTS PC platform.
% 	As the speed of the PCIe is running at 2.5 GHz is very high, they extended the reservation controller to poll the state of the bus at a resolution of one microsecond.
% 	They also reduced the FSB frequency to achieve a typical bandwidth value for embedded platforms.
% 	The synthetic benchmark consists of four real-time flows competing for main memory.
% 	The tasks' periods are harmonic, and the total utilization is 100\%.
% 	So the task set is schedulable under RM.
% 	First, they execute this task set on an unmodified system.
% 	One of the tasks misses its deadline at a near-critical instant.
% 	Once the real-time I/O management is employed, no deadline misses are observed.
% \item What other work is listed as future work?\\
% 	In future work, they want to distinguish traffic from the same peripheral, i.e. giving different priorities to different traffic from the same peripheral. They also plan to provide hardware-based preprocessing and filtering on the real-time bridge to drop less important packets if the main CPU is overloaded.
% \end{itemize}
% Additional questions:
% \begin{itemize}
%   \item What are the limitations/assumptions of this work?\\
%   	Requires redevelopment of drivers for peripherals on host and real-time bridge side.
%   \item Which parts of a system/design process are modified by this work? (e.g. hardware (which feature?), WCET analysis, scheduling, compiler, programming language, \ldots)\\
%   	Hardware: real-time bridges and reservation controller. Software: drivers for bridges and hosts.
% \end{itemize}

\label{sec:related_arch_mod}
