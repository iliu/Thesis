\section{Summary of Results}

This is my summary

%temporally isolation can be relaxed, but must be controlled
It is important to understand that we are not proclaiming that all dynamic behavior in systems are harmful. 
But only by achieving predictability in the hardware architecture can we begin to reason about more dynamic behavior in software.
For example, we discussed that dynamically scheduling threads in hardware causes timing interference. 
However, it is not the switching of threads that is unpredictable, but how the thread switching is triggered that makes it predictable.   
For example, the Giotto\todo{cite} programming model specifies a periodic software execution model that can contain multiple program states. 
If such a programming model was implemented on a thread-interleaved pipeline, different program states might map different tasks to threads or have different number of threads executing within the pipeline.
But by explicitly controller the thread switches in software, the execution time variances introduced is transparent at the software level, allowing potential for timing analysis.


\section{Publications}

\section{Future Work}

Here is what you can keep doing

Talk about future research challenges for a predictable architecture.
\begin{itemize}
  \item synchronization of threads, atomic primitives and memory barrier?
  \item Bus and I/O architectures
\end{itemize}
