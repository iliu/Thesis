In order to propel the use of CPS for safety critical systems, we contend that changes must be made to conventional abstraction layers to introduce ``time'' as its first class citizen.
In this thesis we focused on the ISA abstraction layer and below.  
We explored instruction extensions to the ARM ISA in order to bring temporal semantics to the programs from the ISA.  
We also presented the precision timed ARM (PTARM), an implementation of a PRET machine, in order to provide a timing predictable and composable platform for deterministic execution time.  

To bring temporal semantics to the ISA abstraction layer, we presented a few instruction extensions to the existing instruction set. 
The instructions operate on a platform clock that is synchronous with the execution of instructions. 
The instructions extensions allow programmers to specify the timing properties of program segments, and throw hardware exceptions when the timing specifications are not met.
In this way, our instruction extensions does not over constrain the temporal semantics of the ISA, and still allows architecture innovation to improve program performance. 
These extensions allows programmers to begin reasoning about temporal properties of the programs independent of the underlying execution platform, provided that the ISA is faithfully implemented.

PTARM exploits thread-level parallelism for performance by employing a predictable thread-interleaved pipeline. 
This removes the unpredictability when handling pipeline hazards, and provides temporal isolation for all hardware threads within the pipeline.
PTARM uses scratchpads instead of caches to expose the memory hierarchy, which enables simpler and tighter WCET analysis. 
With a bank privatized DRAM controller, PTARM gives predictable memory access latencies to the DRAM for each hardware thread, and preserves temporal isolation for each hardware thread that accesses the DRAM as a shared resource.
We show that PTARM offers both timing predictability and composability, equipping CPS platforms the tools to interact with physical processes deterministically.

We also demonstrated the the benefits of a PRET machine in the context of a real-time engine fuel rail simulator and embedded security.
For simulating engine fuel rails in real time, we showed a platform that uses multiple PTARM cores which communicate through local shared buffers. 
The predictable timing of PTARM enables a software timing based synchronization between the cores that is implemented with the timing instruction extensions.  
In the context of embedded security, the underlying architecture implementing encryption algorithms are susceptible to timing side-channel attacks, which allows attackers to exploit the uncontrollable execution time variance to derive the key. 
We show that with a predictable architecture and controllable timing properties of the program, we not only defend against all timing side-channel attacks, but eliminate its root cause. 


Several open challenges and questions were mentioned in this thesis, which provides grounds for future research opportunities.  
First, we continue to explore the formalization of the timing extensions to the ISA. 
The introduction of temporal semantics in the ISA should be platform independent; our implementation in PTARM merely open up opportunities for further experimentation and research. 
Nailing down the formal semantics of each extension is key to a consistent meaning of ``time'' independent of the underlying platform implementation. 
Second, how a predictable pipeline and memory controller handles external interrupts and I/O devices still remains an open question.
With the plethora of complex interfaces and protocols for modern high speed I/O interactions, typical I/O controllers are implemented in hardware.
However, we envision a predictable architecture with precise timing control to enable software implementations of protocols typically implemented in hardware. 
This can enable flexibility and more efficient design effort, leading to faster time-to-market and more feature rich designs.        
Third, the interfacing with a timing predictable bus or interconnect can give means to implementing a predictable multicore architecture.
We showed a primitive multicore implementation of PRET cores when simulating engine fuel rails in real time.
However, as communication schemes get more complex, how synchronization and communication of multiple PRET cores is done remains an open question and research challenge. 

It is important to understand that we are not proclaiming that all dynamic behavior in systems are harmful.
However, the dynamic behavior must be controllable and predictable. 
For example, dynamically scheduling hardware threads in the architecture causes uncontrollable timing interference because the triggering of thread switches is hidden from, and cannot be explicitly controlled by, the programmer.
We argue that only by achieving predictability in the architecture and platforms can we begin to reason about more dynamic behavior in software.
With a predictable architecture and the introduction of temporal semantics in the ISA, we hope to provide a deterministic foundation to enable larger and more efficient designs of cyber-physical systems. 

