In order to improve the efficiency and scalability of the design of CPS and safety critical systems, we contend that changes must be made to conventional abstraction layers to introduce \emph{time} as a first class citizen.
In this thesis we focus on doing so for the ISA abstraction layer and below.  
We explore instruction extensions to the ARM ISA to bring temporal semantics to the program, independent of architecture implementation.  
We also present the precision timed ARM (PTARM), an implementation of a PRET machine, in order to provide a timing predictable and composable platform for deterministic execution times.  

To bring temporal semantics to the ISA abstraction layer, we present a few instruction extensions to the existing instruction set. 
The instructions operate on a platform clock that is synchronous with the execution of instructions. 
The instruction set extensions allow programmers to specify the timing properties of program segments, and to throw hardware exceptions when the timing specifications are not met.
In this way, our instruction extensions do not over constrain the temporal semantics of the ISA, and continue to allow architecture innovation to improve program performance. 
These extensions allow programmers to begin reasoning about temporal properties of the programs independent of the underlying execution platform, provided that the ISA is faithfully implemented.

The PTARM exploits thread-level parallelism for performance by employing a predictable thread-interleaved pipeline. 
This removes the unpredictability when handling pipeline hazards, and provides temporal isolation for all hardware threads within the pipeline.
PTARM uses scratchpads instead of caches to expose the memory hierarchy, which enables a simpler and more precise WCET analysis of memory accesses.  
With a bank privatized DRAM controller, PTARM provides predictable DRAM access latencies for each hardware thread, and preserves temporal isolation between the hardware threads that access the DRAM as a shared resource.
The timing predictability and composability provided by PTARM does not come at the cost of aggregate performance, as our benchmarks show an improved throughput for both the pipeline and DRAM memory controller when fully utilized.  
Although achieving full utilization requires that applications have sufficient concurrency, the deterministic architecture can better equip CPS platforms with the ability to handle the concurrency and the uncontrollable timing properties exhibited by physical processes.

We also demonstrate the benefits of a PRET machine in the context of a real-time engine fuel rail simulator and embedded security.
To simulate an engine fuel rail in real time, we implement a platform that uses multiple PTARM cores that communicate through local shared buffers. 
The predictable timing of PTARM allows us to statically verify that the timing constraints are met.  
The timing control provided by the extended ISA enables us to implement a software based low overhead time synchronized communication protocol between the hardware threads, saving the resources required to implement a full hardware interconnection system.
These features of PTARM enable us to implement a scalable solution that can simulate, in real time, a 237 node common fuel rail systems in a single Xinlinx Virtex 6 FPGA. 
In the context of embedded security, the underlying architecture implementing encryption algorithms is susceptible to timing side-channel attacks.
Attackers can exploit the uncontrollable execution time variances caused by the architecture or algorithm to derive the key. 
We implement the RSA and DSA encryption algorithms on a PRET architecture, and show that a predictable architecture with controllable timing properties in the ISA not only defends against all timing related side-channel attacks, but eliminates the root cause of them. 

We continue to investigate the several challenges and questions mentioned in this thesis.
First, we continue to explore the formalization of the timing extensions to the ISA. 
The introduction of temporal semantics in the ISA should be platform independent; our implementation in PTARM merely opens up opportunities for further experimentation and research. 
Nailing down the formal semantics of each instruction extension is key to a consistent meaning of ``time'' that is independent of the underlying hardware implementation. 
Second, we continue to experiment with how a predictable pipeline and memory controller handles external interrupts and I/O devices.     
With the plethora of complex interfaces and protocols for modern high speed I/O interactions, typical I/O controllers are implemented in hardware.
However, we envision a predictable architecture with precise timing control can enable software implementations of protocols typically implemented in hardware due to the lack of precise control over timing in software.
A software implementation can enable flexibility for different protocols and reduce design effort, leading to faster time-to-market and more feature rich designs.        
Third, we continue to explore the interfacing with a timing predictable bus or interconnect, which can be used in timing predictable multicore systems.
In our real-time engine fuel rail simulator, we show a multicore implementation with PRET architectures that uses local shared memories for communication and  a timing based synchronization communication protocol implemented in software.  
However, as communication schemes and applications get more complex, the interconnect or bus will play a more integral role in the connection and communication of multiple PRET cores.
Thus, our future work also includes predictable communication protocols across interconnects and shared buses that leverage the predictable timing of the PRET architecture.     

It is important to understand that we are not proclaiming that all dynamic behavior in systems are harmful.
However, the dynamic behavior must be controllable and predictable. 
For example, dynamically scheduling hardware threads in the architecture causes uncontrollable timing interference because the triggering of thread switches is hidden from, and cannot be explicitly controlled by, the programmer.
We argue that only by achieving predictability in the architecture and platforms can we begin to reason about more dynamic behavior in software.
With a predictable architecture and the introduction of temporal semantics in the ISA, we hope to provide a timing deterministic foundation in the lower levels of abstraction.
In doing so, we enable larger and more complex designs of cyber-physical systems to gain more precise and efficient control over the timing properties of the system.  

