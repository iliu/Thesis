In this chapter we present the design principles of a PREcision Timed (PRET) Machine.
Specifically, we discuss the implementation of a predictable pipeline and memory controller, and present timing extensions to the ISA. 
It is important to understand why and how current architectures fall short of timing predictability and repeatability.
Thus, we first discuss the common architectural designs and their effects on execution time, and point out some key issues and trade-offs when designing architectures for predictable and repeatable timing.

\section{Pipelines}
The introduction of pipelining vastly improved the performance of processors.
Pipelining increases the number of instructions that can be processed at one time by splitting up instruction execution into multiple steps.
It allows for faster clock speeds, and improves instruction throughput compared to single cycle architectures.
Ideally each in processor cycle, one instruction completes and leaves the pipeline as another enters and begins execution. 
In reality, different pipeline hazards occur which reduce the throughput and create stalls in the pipeline.
The techniques introduced to mitigate the penalties of pipeline hazards greatly effect to the timing predictability and repeatability of architectures.     
We analyze several commonly used techniques to reduce the performance penalty from hazards, and show their effects on execution time and predictability. 

\subsection{Pipeline Hazards}
\label{sec:pipeline_hazards}
\subsubsection{Data Hazards}

\begin{wrapfigure}{r}{0.5\textwidth}
  \vspace{-30pt}
  \begin{center}
    \includegraphics[scale=.65]{figs/sample_data_dependent_code}
  \end{center}
  \vspace{-3mm}
  \caption{Sample code with data dependencies}
  \label{fig:sample_data_dependent_code}
\end{wrapfigure}

Data hazards occur when the data needed by an instruction is not yet available.
Pipelines begin the execution of instructions before preceding ones are finished, so consecutive instructions that are data-dependent could simultaneously be executing in the pipeline.
For example, the code in figure~\ref{fig:sample_data_dependent_code} shows assembly instructions from the ARM instruction set architecture (ISA) that each depend on the result of its previous instruction.
Figure~\ref{fig:data_depend_execution_non_interleaved} shows two ways data hazards are commonly handled in pipelines. 

\begin{figure}
\vspace{-20pt} 
\begin{center}
\includegraphics[scale=.6]{figs/data_depend_execution_non_interleaved}
\end{center}
\vspace{-3mm}
\caption{Handling of data dependencies in single threaded pipelines}
\label{fig:data_depend_execution_non_interleaved}
\end{figure}

In the figure, time progresses horizontally towards the right.
Each time step, or column, represents a processor cycle.
Each row represents an instruction that is fetched and executed within the pipeline.
Each block represents the instruction entering the different stages of the pipeline -- fetch (F), decode (D), execute (E), memory (M) and writeback (W).   
We assume a classic five stage RISC pipeline.

A simple but effective technique stalls the pipeline until the previous instruction completes.
This is shown in the top of figure~\ref{fig:data_depend_execution_non_interleaved}, as delays are inserted to wait for the results from previous instructions.
The dependencies between instructions are explicitly shown in the figure to make clear why the pipeline delays are necessary.
The performance penalty incurred in this case comes from the pipeline delays inserted.

\emph{Data forwarding} is commonly used to mitigate the need for inserting delays when data hazards occur.
Pipelines split up the execution of instructions into different execution stages. 
Thus, the results from an instruction could be ready, but waiting to be committed in the last stage of the pipeline.
Data forwarding utilizes this and introduces backwards data paths in the pipeline, so earlier pipeline stages can access the data from instructions in later stages that have not yet committed.
This greatly reduces the amount of delays inserted in the pipeline, as instructions can access the results of previous instructions before they commit.
The circuitry of data forwarding usually consists of the backwards data paths and multiplexers in the earlier pipeline stages to select the correct data to be used.    
The pipeline controller dynamically detects whether a data-dependency exists, and changes the selection bits of the multiplexers accordingly to select the correct operands.

The bottom of figure~\ref{fig:data_depend_execution_non_interleaved} shows the execution sequence of the previous example in a pipeline with data forwarding.
No pipeline delays are inserted for the first \emph{sub} and \emph{ldr} instruction because the data they depend on are forwarded with the forwarding paths.
However, delays are still inserted for the second \emph{sub} instruction after the \emph{ld} instruction.
For longer latency operations, such as memory accesses, the results are not yet available to be forwarded by the forwarding paths, so pipeline delays are still required. 
This illustrates the limitations of data forwarding.
They can address data hazards that result from pipelining, such as read-after-write register operations, but they cannot address data hazards that result from long latency operations, such as memory operations.
More involved techniques such as the out-of-order execution or superscalars are required to mitigate the effects of long latency operations.

The handling of data hazards in pipelines can cause instructions to exhibit dynamic execution times.  
For example, figure~\ref{fig:data_depend_execution_non_interleaved} shows the \emph{sub} instruction, in both top and bottom figures, exhibiting different execution times. 
To determine the execution time of instructions on pipelines that stall for data hazards, we need to determine when a stall is inserted, and how long the pipeline is stalled for.
Stalls are required when the current instruction uses the results of a previous instruction that is still in execution in the pipeline.
Thus, depending on the pipeline depth, a window of previous instructions needs to be checked to determine if any stalls are inserted.     
The length of the stall is determined by the execution time of the dependent instruction, because the pipeline will stall until that instruction completes.
Data forwarding does not remove the data hazards, but only reduces the number of stalls required to take care of the data hazards.  
Thus, to determine the execution time when data forwarding is used, timing analysis needs to determine when the data forwarding circuitry cannot not forward the data for data hazards.
This can similarly be done by observing the window of instructions not yet committed in the pipeline.  
The difference is, instead of all data dependencies, only data dependencies from long latency operations need to be detected.

Both techniques used for handling data hazards caused the execution time of instructions to depend on a window of previous instructions.
The deeper the pipeline, the larger the window of instructions execution time will depend on. 
Thus, static execution time analysis needs to model and account for this additional the window of instructions on pipelined architectures that use stalling or forwarding to handle the data hazards. 

\subsubsection{Control Hazards}
\begin{wrapfigure}{r}{0.5\textwidth}
  \vspace{-20pt}
  \begin{center}
    \includegraphics[scale=.65]{figs/sample_gcd_code}
  \end{center}
  \vspace{-3mm}
  \caption{GCD with conditional branches}
  \label{fig:sample_gcd_code}
\end{wrapfigure}

Branches cause control-flow hazards (or control hazards) in the pipeline; the instruction after the branch, which should be fetched the next cycle, is unknown until after the branch instruction is completed.
Conditional branches further complicates matters, as whether or not the branch is taken depends on an additional condition that could also be unknown when the conditional branch is in execution. 
The code segment in figure~\ref{fig:sample_gcd_code} implements the \emph{Greatest Common Divisor} (GCD) algorithm using the conditional branch instructions \emph{beq} (branch equal) and \emph{blt} (branch less than) in the ARM ISA.  
Conditional branch instructions in ARM branch based on conditional bits that are stored in a processor state register.
Those conditional bits can be set based on the results of standard arithmetic instructions \todo{cite arm manual}.
The \emph{cmp} instruction is one such instruction that subtracts two registers and sets the conditional bits according to the results.
The GCD implementation shown in the code uses this mechanism to determine whether to continue or end the algorithm.
Figure~\ref{fig:branch_execution_non_interleaved_pipeline} shows the execution of the conditional branches from our example, and demonstrates two commonly used techniques to handling control hazards in pipelines. 
To show only the timing effects of handling control hazards, we assume an architecture with data forwarding that handles data hazards.
As there are no long latency instructions in our example, all stalls observed in the figure are caused by the handling of control hazards.  

\begin{figure}
\begin{center}
\noindent\makebox[\textwidth]{%
\includegraphics[scale=.58]{figs/branch_execution_non_interleaved_pipeline}}
\end{center}
\vspace{-3mm}
\caption{Handling of conditional branches in single threaded pipelines}
\label{fig:branch_execution_non_interleaved_pipeline}
\end{figure}

Similar to data hazards, control hazards can also be handled by stalling the pipeline until the branch instruction completes. 
This is shown on the left of figure~\ref{fig:branch_execution_non_interleaved_pipeline}. 
%The dependencies between instructions are explicitly shown to make clear why the pipeline delays are necessary.
Branch instructions typically calculate the target address in the execute stage, so two pipeline delays are inserted before the fetching of the \emph{blt} instruction, to wait for \emph{beq} to complete the target address calculation. 
The reasoning applies to the two pipeline delays inserted before the \emph{sub} instruction. 
The performance penalty (often referred to as the \emph{branch penalty}) incurred in this case is the two delays inserted after every branch instruction, to wait for the branch address calculation to complete.

To mitigate the branch penalty, some architectures enforce the compiler to insert one or more non-dependent instructions after each branch instruction.
These instruction slots are called branch delay slots, and are always executed before the pipeline branches to the target address. 
This way, instead of wasting cycles to wait for the target address calculation, the pipeline continues to execute useful instructions before it branches.
However, if the compiler cannot place useful instructions in the branch delay slot, \emph{nops} need to be inserted into those slots to ensure correct program execution.
Thus, branch delay slots are less effective for deeper pipelines because more delay slots need to be filled by the compiler to account for the branch penalty.
  
In attempt to remove the need of inserting pipeline bubbles, \emph{branch predictors} were invented to predict the results of a branch before it is resolved\todo{citation}.
%Branch predictors have been heavily researched.
Many clever branch predictors have been proposed, and they can accurately predict branches up to 93.5\%\todo{citation}.
%With branch predictor, the pipeline fetches the next instruction based upon the results of the branch prediction, and continues to execute speculatively.
Branch predictors predict the condition and target addresses of branches, so pipelines can speculatively continue execution based upon the prediction.  
If the prediction was correct, no penalty occurs for the branch, and execution simply continues. 
However, when a mispredict occurs, then the speculatively executed instructions need to be flushed and the correct instructions need to be refetched into the pipeline for execution.
The right of figure~\ref{fig:branch_execution_non_interleaved_pipeline} shows the execution of GCD in the case of a branch misprediction.
After the \emph{beq} instruction, the branch is predicted to be taken, and the \emph{add} and \emph{mov} instructions from the label \emph{end} is directly fetched into execution. 
When the \emph{cmp} instruction is completed, a misprediction is detected, so the \emph{add} and \emph{mov} instruction are flushed out of the pipeline while the correct instruction \emph{blt} is immediately re-fetched and execution continues.
The misprediction penalty is typically the number of stages between fetch and execute, as those cycles are wasted executing instructions from an incorrect execution path.
This penalty only occurs on a mispredict, thus branch prediction typically yields better average performance and is preferred for modern architectures.
%However, for more complex architectures with caches or other hardware states, the effects of incorrectly fetched instructions on the state of the processor less well-known and studied. 
Nonetheless, it is important to understand the effects of branch prediction on execution time. 

Typical branch predictors predict branches based upon the history of previous branches encountered.  
As each branch instruction is resolved, the internal state of the predictor, which stores the branch histories, is updated and used to predict the next branch.
This implicitly creates a dependency between branch instructions and their execution history, as the prediction is affected by its history.
In other words, the execution time of a branch instruction will depend on the branch results of previous branch instructions.
During static execution timing analysis, the state of the branch predictor is unknown because is it often infeasible to keep track of execution history so far back.   
There has been work on explicitly modeling branch predictors for execution time analysis\todo{citation}, but the results are \todo{the results of branch predictor modeling for execution time analysis}.
The analysis needs to conservatively account for the potential branch mispredict penalty for each branch, which leads to overestimated execution times.
To make matters worse, as architectures grow in complexity, more internal states exist in architectures that could be affected by the speculative execution. 
For example, cache lines could be evicted when speculatively executing instructions from a mispredicted path, changing the state of the cache.  
%When instructions from the correct execution path are re-fetched at branch resolution, a cache miss could be resulted from the change in cache because of the branch misprediction.    
This makes a tight static execution time analysis extremely difficult, if not impossible; explicitly modeling all hardware states and their effects together often lead to an infeasible explosion in state space. 
On the other hand, although the simple method of inserting pipeline bubbles for branches could lead to more branch penalties, the static timing analysis is precise and straight forward, as no prediction and speculative execution occur. 
The timing analysis simply adds the branch penalty to the instruction after a branch. 
Additional penalties from a conditional branch can be accounted for by simply checking for instructions that modify the conditional flag above the conditional branch.
We explicitly showed this simple method of handling branches to point out an important trade-off between speculative execution for better average performance and consistent stalling for better predictability.
Average-case performance can be improved by speculation at the cost of predictability and potentially prolonging the worst-case performance.  
The challenge remains to maintain predictability while improving worst-case performance, and how pipeline hazards are handled play an integral part of tackling this challenge.           

\subsubsection{Structural Hazards}

\subsection{Pipeline Multithreading}
 Multithreaded architectures were introduced to improve instruction throughput over instruction latency.
The architecture optimizes thread-level parallelism over instruction-level parallelism to improve performance.
Multiple hardware threads are introduced into the pipeline to fully utilize thread-level parallelism. 
When one hardware thread is stalled, another hardware thread can be fetched into the pipeline for execution to avoid stalling the whole pipeline. 
To lower the context switching overhead, the pipeline contains physically separate copies of hardware thread states, such as registers files and program counters etc, for each hardware thread.
\begin{figure}
\begin{center}
\includegraphics[scale=.8]{figs/multithreaded_pipeline_block}
\end{center}
\vspace{-30pt}
\caption{Simple Multithreaded Pipeline}
\label{fig:multi-thread pipeline simplified}
\end{figure}
Figure~\ref{fig:multi-thread pipeline simplified} shows a architectural level view of a simple multithreaded pipeline.
It contains 5 hardware threads, so it has 5 copies of the Program Counter (PC) and Register files.
Once a hardware thread is executing in the pipeline, its corresponding thread state can be selected by signaling the correct selection bits to the multiplexers.
The rest of the pipeline remains similar to a traditional 5 stage pipeline as introduced in Hennessy and Pattern\todo{citation}.   
The extra copies of the thread state and the multiplexers used to select them thus contribute to most of the hardware additions needed to implement hardware multithreading.

%The selection of threads for execution is one of the most important factors to fully utilize thread-level parallelism.
%If a thread is stalled waiting for memory access but gets selected to execute in the pipeline, then that instruction slot is wasted and the processor isn't fully utilized.
Ungerer et al.~\cite{Ungerer:2003:survey_multithreading} surveyed different multithreaded architectures and categorized them based upon the \todo{thread selection?} policy and the execution width of the pipeline.
The thread selection policy is the context switching scheme used to determine which threads are executing, and how often a context switch occurs.  
Coarse-grain policies manage hardware threads similar to the way operation systems manage software threads.
A hardware thread gain access to the pipeline and continues to execute until a context switch is triggered.
Context switches occur less frequently via this policy, so less hardware threads are required to fully utilize the processor.
Different coarse-grain policies trigger context switches with different events. 
Some trigger on dynamic events, such as cache miss or interrupts, and some trigger on static events, such as specialized instructions.
Fine-grain policies switch context much more frequently -- usually every processor cycle.
Both coarse-grain and fine-grain policies can also have different hardware thread scheduling algorithms that are implemented in a hardware thread scheduling controller to determine which hardware thread is switched into execution.  
The width of the pipeline refers to the number of instructions that can be fetched into execution in one cycle. 
For example, superscalar architectures have redundant functional units, such as multipliers and ALUs, and can dispatch multiple instructions into execution in a single cycle. 
Multithreaded architectures with pipeline widths of more than one, such as Sumultanous Multithreaded (SMT) architectures, can fetch and execute instructions from several hardware threads in the same cycle.

Multithreaded architectures typically bring additional challenges to execution time analysis of software running on them.
Any timing analysis for code running on a particular hardware thread needs to take into account not only the code itself, but also the thread selection policy of the architecture and sometimes even the execution context of code running on other hardware threads.
For example, if dynamic coarse-grain multithreading is used, then a context switch could occur at any point when a hardware thread is executing in the pipeline.
This not only has an effect on the control flow of execution, but also the state of any hardware that is shared, such as caches or branch predictors.    
Thus, it becomes nearly impossible to estimate execution time without knowing the exact execution state of other hardware threads and the state of the thread scheduling controller.
However, it is possible to for multithreaded architectures to fully utilize thread-level parallelism while still maintaining timing predictability.
Thread-interleaved pipelines use a fine-grain thread switching policy with round robin thread scheduling to achieve high instruction throughput while still allowing precise timing analysis for code running on its hardware threads. 
Below, its architecture and trade-offs are described and discussed in detail along with examples and explanation of how timing predictability is maintained.
Through the remainder of this chapter, we will use the term ``thread'' to refer to explicit hardware threads that have physically separate register files, program counters, and other thread states.
This is not to be confused the common notion of ``threads'', which is assumed to be software threads that is managed by operating systems with thread states stored in memory.

\subsection{Thread-Interleaved Pipelines}
\label{section:pret_thread_pipeline}
%\todo{Go through and make sure you don't say minimum threads as pipeline stages is a requirement, since later we have 4 threads in a give stage pipeine.}
\begin{figure}
  \vspace{-20pt}
  \begin{center}
    \includegraphics[scale=.55]{figs/thread-interleaved-execution}
  \end{center}
  \vspace{-20pt}
  \caption{Sample execution sequence of a thread-interleaved pipeline with 5 threads and 5 pipeline stages}
  \label{fig:execution_thread_interleaved_pipeline}
\end{figure}
The thread-interleaved pipeline was introduced to improve the response time of handling multiple I/O devices \todo{citation}.
I/O operations often stall from the communication with the I/O devices.
Thus, interacting with multiple I/O devices leads to wasted processor cycles that are idle waiting for the I/O device to respond.
By employing multiple hardware thread contexts, a hardware thread stalled from the I/O operations does not stall the whole pipeline, as other hardware threads can be fetched and executed.
% In a thread-interleaved pipeline, a thread context switch occurs every processor cycle, and the threads are cycled through in a round robin fashion. 
% This ensures each thread gets equal access to the process resource, so threads that aren't stalled are guaranteed to make progress.
Thread-interleaved pipelines use fine-grain multithreading; every cycle a context switch occurs and a different hardware thread is fetched into execution. 
The threads are scheduled in a deterministic round robin fashion. 
This also reduces the context switch overhead down to nearly zero, as no time is needed to determine which thread to fetch next.
Barely any hardware is required to implement round robin thread scheduling; a simple $log(n)$ bit up counter (for $n$ threads) would suffice.         
Figure~\ref{fig:execution_thread_interleaved_pipeline} shows an example execution sequence from a 5 stage thread-interleaved pipeline with 5 threads.
The thread-interleaved pipelines shown and presented in this thesis are all of single width.
The same code segments from figure~\ref{fig:sample_gcd_code} and figure~\ref{fig:sample_data_dependent_code} are being executed in this pipeline. 
Threads 0, 2 and 4 execute GCD (figure~\ref{fig:sample_gcd_code}) and threads 1 and 3 execute the data dependent code segment (figure~\ref{fig:sample_data_dependent_code}).
Each hardware thread executes as an independent context and their progress is shown in figure~\ref{fig:execution_thread_interleaved_pipeline} with thick arrows pointing to the execution location of each thread at t0.
We can observe from the figure that each time step an instruction from a different hardware thread is fetched into execution and the hardware threads are fetched in a round robin order.
At time step 4 we begin to visually see that each time step, each pipeline stage is occupied by a different hardware thread.
The fine-grained thread interleaving and the round robin scheduling combine to form this important property of thread-interleaved pipelines, which provides the basis for a timing predictable architecture design.

For thread-interleaved pipelines, if there are enough thread contexts, for example -- the same number of threads as there are pipeline stages, then at each time step no dependency exists between the pipeline stages since they are each executing on a different thread. 
As a result, data and control pipeline hazards, the results of dependencies between stages within the pipelines, no longer exist in the thread-interleaved pipeline.    
We've already shown from figure~\ref{fig:branch_execution_non_interleaved_pipeline} that when executing the GCD code segment on a single-threaded pipeline, control hazards stem from branch instructions because of the address calculation for the instruction after the branch.
However, in a thread-interleaved pipeline, the instruction after the branch from the same thread is not fetched into the pipeline until the branch instruction is committed.
Before that time, instructions from other threads are fetched so the pipeline is not stalled, but simply executing other thread contexts.
This can be seen in figure~\ref{fig:execution_thread_interleaved_pipeline} for thread 0, which is represented with instructions with white backgrounds.
The \emph{cmp} instructions, which determines whether next conditional branch \emph{beq} is taken or not, completes before the \emph{beq} is fetched at time step 5.
The \emph{blt} instruction from thread 0, fetched at time step 10, also causes no hazard because the \emph{beq} is completed before \emph{blt} is fetched.
The code in figure~\ref{fig:sample_data_dependent_code} is executed on thread 1 of the thread interleave pipeline in figure~\ref{fig:execution_thread_interleaved_pipeline}.
The pipeline stalls inserted from top of figure~\ref{fig:data_depend_execution_non_interleaved} are no longer needed even without a forwarding circuitry because the data-dependent instructions are fetched after the completion of its previous instruction.
In fact, no instruction in the pipeline is dependent on another because each pipeline stage is executing on a separate hardware thread context.
Therefore, the pipeline does not need to include any extra logic or hardware for handling data and control hazards in the pipeline. 
This gives thread-interleaved pipelines the advantage of a simpler pipeline design that requires less hardware logic, which in turns allows the pipeline clock speed to increase.
Thread-interleaved pipelines can be clocked at higher speeds since each pipeline stage contains significantly less logic needed to handle hazards.
The registers and processor states use much more compact memory cells compared to the logic and muxes used to select and handle hazards, so the size footprint of thread-interleaved pipelines are also typically smaller.

For operations that have long latencies, such as memory operations or floating point operations, thread-interleaved pipelines hides the latency with its execution of other threads. 
Thread 3 in figure~\ref{fig:execution_thread_interleaved_pipeline} shows the execution of a \emph{ld} instruction that takes the same 5 cycles as shown in figure~\ref{fig:data_depend_execution_non_interleaved}.
We again assume that this \emph{ld} instruction accesses data from the main memory. 
While the \emph{ld} instruction is waiting for memory access to complete, the thread-interleaved pipeline executes instructions from other threads.
The next instruction from thread 3 that is fetched into the pipeline is again the same \emph{ld} instruction.  
As memory completes its execution during the execution of instructions from other threads, we replay the same instruction to pick up the results from memory and write it into registers to complete the execution of the \emph{ld} instruction. 
It is possible to directly write the results back into the register file when the memory operation completes, without cycling the same instruction to pick up the results.
This would require hardware additions to support and manage multiple write-back paths in the pipeline, and a multi write ported register file, so contention can be avoided with the existing executing threads.
In our design we simply replay the instruction for write-backs to simplify design and piggy back on the existing write-back datapath.
%We showed a memory access instruction in our example, but the same reasoning is applied to floating point instructions or any long latency instruction.  
Multithreaded pipelines typically mark threads inactive when they are waiting for long latency operations.
Inactive threads are not fetched into the pipeline, since they cannot make progress even if they are scheduled. 
This allows the processor to maximize throughput by allowing other threads to utilize the idle processor cycles.   
However, doing so has non-trivial effects on thread-interleaved pipelines and the timing of other threads. 

First, if the number of ``active'' threads falls below the number of pipeline stages, then pipeline hazards are reintroduced; it is now possible for the pipeline to be executing two instructions from the same thread that depend on each other simultaneously. 
This can be circumvented by inserting pipeline bubbles when there aren't enough active threads. 
For example, as shown in figure~\ref{fig:three_thread_pipeline}, for our 5 stage thread-interleaved pipeline that has 5 threads, if two threads are waiting for main memory access and are marked inactive, then we insert 2 NOPs every round of the round-robin schedule to ensure that no two instructions from the same thread exists in the pipeline.
Note that if the 5 stage thread-interleaved pipeline contained 7 threads, then even if 2 threads are waiting for memory, no NOP insertion would be needed since instructions in each pipeline stage in one cycle would still be from a different thread.   
NOP insertions only need to occur when the number of active threads drops below the number of pipeline stages.   
\begin{figure}
  \vspace{-20pt}
  \begin{center}
    \includegraphics[scale=.6]{figs/three_thread_pipeline}
  \end{center}
  \vspace{-20pt}
  \caption{Execution of 5 threads thread-interleaved pipeline when 2 threads are inactive}
  \label{fig:three_thread_pipeline}
\end{figure}

The more problematic issue with setting threads inactive whenever long latency operations occur is the effect on the execution frequencies of other threads in the pipeline.
When threads are scheduled and unscheduled dynamically, the other threads in the pipeline would dynamically execute more or less frequently depending on how many threads are active.
This complicates timing analysis since the thread frequency of one thread now depends on the program state of all other threads. 
In order for multithreaded architectures to achieve predictable performance, \emph{temporal isolation} must exist in the hardware between the threads.
Temporal isolation is the isolation of timing behaviors of a thread from other thread contexts in the architecture.
With temporal isolation, the timing analysis is greatly simplified, as software running on individual threads can be analyzed separately without worry about the effects of integration. 
If temporal isolation is broken, any timing analysis needs to model and explore all possible combinations of program state of all threads, which is typically infeasible.
The round-robin thread scheduling of thread-interleaved pipelines is a way of achieving temporal isolation for a multithreaded architecture.
Unlike coarse-grain dynamically switched multithreaded architectures, thread-interleaved pipelines can maintain the same round-robin thread schedule despite the execution context of each thread within the pipeline. 
This is a step towards achieving temporal isolation amongst the threads, as the execution frequency of threads does not change dynamically.  
However, dynamically scheduling and unscheduling threads based upon long-latency operations breaks temporal isolation amongst threads.
Thus, our thread-interleaved pipeline does not mark threads inactive on long latency operations, but simply replays the instruction whenever the thread is fetched.  
Although this slightly reduces the utilization of the thread-interleaved pipeline, but threads are decoupled and timing analysis can be done individually for each thread without interference from other threads.
At the same time, we still preserve most of the benefits of latency hiding, as other threads are still executing during the long latency operation.

%TODO: exception handling
\todo{talk about xmos handling exceptions and our handling of exceptions here}

Shared hardware units within multithreaded architectures could also easily break temporal isolation amongst the threads. 
Two main issues arise when a hardware unit is shared between the threads. 
The first issue arises when shared hardware units share the same state between all threads. 
If the state of hardware unit is shared and can be modified by any thread, then it is nearly impossible to get a consistent view of the hardware state from a single thread during timing analysis.
Shared branch predictors and caches are prime examples of how a shared hardware state can cause timing inference between threads.
If a multithreaded architecture shares a branch predictor for all threads, then the branch table entries can be overwritten by branches from any thread.
This means that each thread's branches can cause a branch mispredict for any other thread. 
Caches are especially troublesome when shared between threads in a multithreaded architecture. 
Not only does it make the execution time analysis substantially more difficult, it also decreases overall performance for each thread due to cache thrashing, an event where threads continuously evict each other threads cache lines in the cache\todo{citation}.
To achieve temporal isolation between the threads, the hardware units in the architecture must not share state between the threads. 
Each thread must have its own consistent view of the hardware unit states, without the interference from other threads.
For example, each thread in our thread-interleaved pipeline contains its own private copy of the registers and thread states.    
We already showed why thread-interleaved pipelines do not need branch predictors because they remove control-hazards, and we will discuss a timing predictable memory hierarchy that uses scratchpads instead of caches in section~\ref{section:memory_system}.
The sharing of hardware state between threads also increases security risks in multithreaded architectures. 
Side-channel attacks on encryption algorithms\todo{cite} take advantage of the shared hardware states to disrupt and probe the execution time of threads running the encryption algorithm to crack the encryption key.
We will discuss this in detail in section~\ref{sec:app_side_channel_attack} and show how a predictable architecture can prevent timing side-channel attacks for encryption algorithms.   

The second issue that arises is that shared hardware units create structural hazards -- hazards that occur when a hardware unit needs to be used by two or more instructions at the same time.
Structural hazards typically occur in thread-interleaved pipelines when the shared units take longer than one cycle to access. 
The ALU, for example, is shared between the threads.
But because it takes only one cycle to access, there is no contention even when instructions continuously access the ALU in subsequent cycles.
On the other hand, a floating point hardware unit typically takes several cycles to complete its computation.
If two or more threads issue a floating point instruction in subsequent cycles, then contention arises, and the the second request must be queued up until the first request completes its floating point computation.   
This creates timing interference between the threads, because the execution time of a floating point instruction from a particular thread now depends on if other threads are also issuing floating point instructions simultaneously. 
If the hardware unit can be pipelined to accept inputs every processor cycle, then we can remove the the contention caused by the hardware unit, since accesses no longer need to be queued up.
The shared memory system in a thread-interleaved pipeline also creates structural-hazards in the pipeline.
In section~\ref{section:memory_system} we will discuss and present our memory hierarchy along with a redesigned DRAM memory controller that supports pipelined memory accesses.
%The assumption here is still that we have a single issue pipeline architecture.   
If pipelining cannot be achieved, then any timing analysis of that instruction must include a conservative estimation that accounts for thread access interference and contention management.
Several trade-offs need to be considered when deciding how to manage the thread contention to the hardware unit.

A time division multiplex access (TDMA) schedule to the hardware unit can be enforced to decouple the access time of threads remove timing interference.
A TDMA access scheme certainly creates an non-substantial overhead compared to conventional queuing schemes, especially if access to the hardware unit is rare and sparse.
However, in a TDMA scheme, each threads wait time to access the shared resource depends on the time offset in regards to the TDMA schedule, and is decoupled from the accesses of other threads.
Because of that, it is possible obtain a tighter worst case execution time analysis per thread.
For a TDMA scheme, the worst case access time occurs when an access just missed its time slot and must wait a full cycle before accessing the hardware unit.
For a conventional queuing scheme where each requester can only have one outstanding request, the worst case happens when every other requester has a request in queue, and the first request is just beginning to be serviced.   
At first, it may seem that the worst case execution time of a TDMA scheme may seem similar to the basic queuing scheme.
For timing analysis at an unknown state of the program, no assumption can be made on the TDMA schedule, thus the worst case time must be used for conservative estimations. 
However, because the TDMA access schedule is static, and access time is decoupled from other threads, there is potential to obtain tighter timing analysis for accesses by inferring access slot hits and misses for future accesses. 
For example, based upon the execution time offsets of a sequence of accesses to the shared resource, we may be able to conclude that at least one access will hit its TDMA access slot and get access right away.
We can also possibly derive more accurate wait times for the accesses that do not hit its access slots based upon the elapsed time between accesses.
An in depth study of WCET analysis of TDMA access schedules is beyond the scope of the thesis.
But these are possibilities now because there is no timing interference between the threads. 
A queue based mechanism would not be able to achieve better execution time analysis without taking into account the execution context of all other threads in the pipeline.

It is important to understand that we are not proclaiming that all dynamic behavior in systems are harmful. 
But only by achieving predictability in the hardware architecture can we begin to reason about more dynamic behavior in software.
For example, we discussed that dynamically scheduling threads in hardware causes timing interference. 
However, it is not the switching of threads that is unpredictable, but how the thread switching is triggered that makes it predictable.   
For example, the Giotto\todo{cite} programming model specifies a periodic software execution model that can contain multiple program states. 
If such a programming model was implemented on a thread-interleaved pipeline, different program states might map different tasks to threads or have different number of threads executing within the pipeline.
But by explicitly controller the thread switches in software, the execution time variances introduced is transparent at the software level, allowing potential for timing analysis.

%TODO: Talk about the trade-offs of threaded interleaved pipelines to summarize 
In this section we introduced a predictable thread-interleaved pipeline design that provides temporal isolation for all threads in the architecture.
The thread-interleaved pipeline favors throughput over single thread latency, as multiple threads are executed on the pipeline in a round robin fashion.  
We will present in detail our implementation of this thread-interleaved pipeline in chapter~\ref{chapter:ptarm}, and show how the design decisions discussed in this chapter are applied.

\section{Memory System}
\label{section:memory_system}
While pipelines designs continue to improve, memory technology has been struggling to keep up with the increase in clock speed and performance.
Even though memory bandwidth can be improved with more bank parallelization, the memory latency remains the bottle neck to really improving memory performance.
Common memory technologies used in embedded systems contain a significant trade off between access latency and capacity. 
Static Random-Access Memories (SRAM) provides sufficient latency that allows single cycle memory access latencies from the pipeline.
However, the hardware cost to implement each memory cell prohibits large capacities to be implemented close to the processor.
On the other hand, Dynamic Random-Access Memories (DRAM) uses a compact memory cell design that can easily be designed into larger capacity memory blocks.
But the memory cell of DRAMs must be constantly refreshed due to charge leakage, and the large capacity of DRAM cells in a memory block design often prohibit faster access latencies.
%TODO: talk about about embedded systems seldom need disk?
To bridge the latency gap between the pipeline and memory, smaller memories are placed in between the pipeline and larger memories to act as a buffer, forming a memory hierarchy.
The smaller memories give faster access latencies at the cost of lower capacity, while larger memories make up for that with larger capacity but slower access latencies. 
The goal is to speed up program performance by placing commonly accessed values closer to the pipeline, while less access values are placed farther away.

\subsection{Caches vs. Scratchpads}
\begin{wrapfigure}{r}{0.5\textwidth}
  \vspace{-20pt}
  \begin{center}
    \includegraphics[scale=.5]{figs/conventional_mem_hierarchy}
  \end{center}
  \vspace{-20pt}
  \caption{Memory Hierarchy w/ Caches}
  \label{fig:conventional_mem_hierarchy}
  \vspace{-10pt}
\end{wrapfigure}   
A \emph{CPU Cache} (or cache) is commonly used in the memory hierarchy to manage the smaller fast access memory made of SRAMs.
The cache manages of contents of the fast access memory in hardware by leveraging the spatial and temporal locality of data accesses. 
The main benefits of the cache is that it abstracts away the memory hierarchy from the programmer.
When a cache is used, all memory accesses is routed through the cache. 
If the data from the memory access is on the cache, then a cache hit occurs, and the data is returned right away.
However, if data is not on the cache, then a cache miss occurs, and the cache controller fetches the data from the larger memory and adjusts the memory contents on the cache. 
The replacement policy of the cache is used to determine which cache line, the unit of memory replacement on caches, to replace. 
A variety of cache replacement policies have been researched~\todo{cite} and used to optimize for memory access patterns of different applications. 
In fact, modern memory hierarchies often contain multiple layers of hierarchy to balance the trade-off between speed and capacity.
A commonly used memory hierarchy is shown in figure~\ref{fig:conventional_mem_hierarchy}.
If the data value is not found in the L1 cache, then it is searched for in the L2 cache. 
If the L2 cache also misses, then the data is retrieved from main memory, and sent back to the CPU while the L1 and L2 cache updates its contents.
Often times different replacement policies are used at different levels of the memory hierarchy to optimize the hit rate or miss latency of the memory access. 
As sophisticated as this seems, the programmer is oblivious to the different levels of memory hierarchy, and whether or not an access hits the cache or goes all the way out to main memory. 
The memory hierarchy is hidden from the programmer, and the hardware gives its best-effort to optimize memory access latencies without any guarantees.
%talk about cache timing analysis and why it is difficult
%caches also cause other issues - multi threading shared cache, multi core cache coherint

%Introduce scratchpads. uses same technology, but is used for power saving from the circuitry
% Expose the memory hierarchy to the programmer, at the expense of more work from the programmer, but have better control over time 

%discuss possible ways to manage scratchpads, programming models, embedded systems often require in depth anaysis of memory access paterns already etc

%Summarize that the unpredictablility from caches does not stem from the memory technology, but how it is managed by the hardware. 
% That's why using scratchpads exposes the memory hierarchy to the programmer, while still allowing the benefits of smaller, faster memories
% The unpredictability of DRAM technology however stems from the actual memory technology, so we explain the basics of DRAM a little more 
\subsection{DRAM Memory Controller}
% first discuss the basics of DRAM technology, explaining what makes accesses unpredictable

% talk about the back end controller to implement predictable access latencies

% mentions that we will discuss the details and the integration with the PTARM arhictecture in the next section


\section{Instruction Set Architecture Extensions}
\label{sec:programming_models}
Intro text here

\subsection{PRET Programming model Section Header}
\label{sec:pret_prog_model_sec_1}

\subsection{Pret Programming model Section Header 2}
\label{sec:pret_prog_model_sec_2}

Here is another header


